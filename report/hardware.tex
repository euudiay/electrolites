\chapter{Hardware and communications}
\label{ch:hardware}
	\section{Introduction}	

	%%%%%%%%%%%%%%%%%%%%%%%%%%%%%%%%%%%%%%%%%%%%%%%%%%%%%%%%%%%%%%%%%%%%%%%%%%%%%%%
	% ¡FALTA COMENTAR EN ALGUN SITIO QUE EL DESARROLLO SW SE REALIZA EN PARALELO! %
	%%%%%%%%%%%%%%%%%%%%%%%%%%%%%%%%%%%%%%%%%%%%%%%%%%%%%%%%%%%%%%%%%%%%%%%%%%%%%%%


	%que necesidad cubre la parte hardware del proyecto

	% OLD: The hardware in our project cover a main need, an external device to be able to communicate our android device with a *shimmer through 802.15.4. Than device have to be a little and low powered device that can be conected as device through USB in an android device. Little, because a device that disturb a regular work is not valid at all. Low powered, because if the cost of power the divece is higher than use the stack bluetooth we lose an important advantage of use 802.15.4. Able to be coneced throught USB in our android device because this is the only way to interact with android for a external device. And finally able to be connected as device to elimminate the needed of an extra batery that would have incresed the cost and size of our device. \\

	The hardware research and development part of the project objective is covering a main need: production of an external device that enables communication between an Android system and a *shimmer through 802.15.4. Such a device should be a portable and low-powered device that can be plugged via the widely used USB On-The-Go (USB OTG) to a host Android system, acting the device as a slave.\\

	It has to be small sized because of the target application environment: a particular who requires constant, in-home, ambulatory monitorization. 
	%In that scenario devices that disturb regular working conditions are not valid at all;
	In that scenario unobtrusive operation is a main need, and usual life style activity modification is to be minimized. And it has to be low-powered, because were the power cost of application higher than that of the Bluetooth technology, a main advantage of 802.15.4 is lost.\\ 

	The ability to communicate through USB is required because, at the time, it is the most low battery consuming method to interact with Android powered platforms for any external device[quote here wlan and bt costs].\\

	And finally it should be able to act as the slave in the USB connection to avoid the need of an extra power source for the device that would increase the cost and size of the product.\\

	% OLD: In order to achieve this ambitious goal, we divide this develop in milestones that will help us to focus our works in more concrete tasks and correctly finish the project. \\

	In order to achieve such goals, and being aware of the substancial amount of research involved in this part of the project, the decision is made to adopt a milestone driven development which simplified scheduling and helped focusing on specific tasks while maintaining a global view of the evolution and the objectives of the project.\\

	\section{Overview}
	% Before of introduce more infromation about our project we have a section of technologies that will be very usefull to understand all this chapter and that will be referenced many times in other sections

	Before diving any further into the development a section describing technologies involved in the research process is presented, as such information will be key to the understanding of the rest of the chapter and will be throughoutly referenced during the exposition.\\

	Then a description of the hardware research and development process is given, followed by detailed explanation of each projected milestone, including objectives pursued, lines of research developed, results of each one and conclusions obtained. Estimation based decision making being crucial for the correct outcome of the project, special care will be put to explain the motivations for each decision made. The chapter concludes with an exposition of the results of the research and subsequent development.

	%\section{Researched Technologies}
	\section{Technology Research}
	%mini introducción diciendo que para la correcta comprensión de está sección surge la necesidad de explicar en mayor o menor profundidad las siguientes tecnologías usadas en algún punto del desarollo del proyecto.

	% The hardware part of this project containts a lot of terms and technologies which is important to know to a correct understanding of the following section but because of it's size we can't explain it whitout a own section because it would become a very heavy doucment and probably lost the reader attencion and the section purpose.

	This part of the project involves a lot of terms and references a number of technologies and concepts which are important to be acquainted with in order to achieve a full understanding of the current chapter. These explanations will not be presented interlaced with the rest of the sections due to the unmanageable size they would acquire leading to a loss of focus which can only act against full comprehension of the exposed content.

		\subsection{Google ADK \& Arduino}
		\label{ssec:Arduino}
			Android Open Accessory Development Kit, referred as ``Google ADK'' or ``ADK'' for now on,
			is a tool set which allows the development of accessories capable of interacting with 
			Android-running devices. It consists of an Arduino board (particularly MegaADK, which is based
			on the ATmega2560 board) and a series of libraries for interacting with an optional set of
			hardware add-ons (``shields''). There are other compatible kits with different boards, like
			NXP-based mbed \cite{mbed}, yet Arduino MegaADK is the one used here.\\
			
			The Google ADK functioning is actually very simple. Regarding the software, it needs of 
			the developing of both an Android application with accessory communication and the board's
			firmware, which models the behaviour of the accessory. Once the board is flashed with its
			firmware, it has to be connected to a power source (through a dedicated wire or Type-B USB)
			and the Android device. If everything is correct, the latter detects the former and, therefore, 
			the accessory starts its operation.\\
			
			Typically, the capabilities of the Arduino board are broaden with the addition of extra shields,
			so that way it can make use of external sensors and agents. The board has multiple input and
			output pins at the developer's disposal as well, so unforeseen behaviours can be achieved.\\
			
		\subsection{MSP430}
		\label{ssec:msp430}
			MSP430 refers to an entire family of 16-bit CPU microcontrollers from Texas Instruments \cite{msp430}.
			Its most relevant features are:
			\begin{itemize}
				\item \textbf{Very low power consumption:} its several low-power modes make the MSP430 
					family specially suitable for developing embedded systems. Moreover, its brief wakeup 
					transitions from this operating modes are also noteworthy, since these lapses stay 
					around the microsecond ($\mu$s) range.
				\item \textbf{RISC-based Architecture:} its instruction set is particularly narrow \cite{msp430iss}, 
					and thus simple. Reduced instruction sets ease programming when compared to complex ones,
					yet this advantage is not too decisive because of the reason presented next.
				\item \textbf{Simple programming:} Most family members are programmable trough JTAG, which
					makes the debugging process less difficult along with the possibility of swapping
					assembly for C.
				\item \textbf{Wide support and resources:} Texas Instruments provides code examples, software 
					IDEs and thorough documentation as well as it offers extra developing tools like training 
					boards.
			\end{itemize}
			However, it also suffers from some lacks. For instance, MSP430 devices are not equipped with
			external memory bus, what makes flash memory and RAM extensions impossible. In particular, the
			MSP430F66xx family, which the device used within this project belongs to, is provided with only 
			128-256KB of flash storage and 16KB	of RAM --with an extra of 2KB whenever it is not using USB 
			\cite[p. 2]{msp430f6638ds}--. This amount of RAM may be too limited for certain usages (not for 
			this project's particular requirements, though).\\
			
			Concretely, the microcontrollers and boards used in this project are:
			\begin{itemize}
				\item \textbf{Microcontrollers:} \emph{MSP430F5438A} vs. \emph{MSP430F6638}\\
					Both of them are characterized by their low power consumption as well as their voltage 
					range, from 1.8V to 3.6V. Although they share most of their specs, there are some points
					where they differ:
					\begin{enumerate}
						\item \textbf{System Clock Frequency:} in the case of the MSP430F5438A microcontroller
							it is up to 25MHz, whereas MSP430F6638's clock only reaches 20MHz.
						\item \textbf{Wakeup time lapse:} meanwhile TI claims MSP430F5438A wakes from 
							standby mode in less than 5$\mu$s, MSP430F6638 needs less than 3$\mu$s.
					\end{enumerate}
					However, the most important difference lies in the USB support the second microcontroller
					provides, which is the reason why it was chosen for the accessory development. At any rate, 
					full technical specification can be found at \cite{5438a} and \cite{msp430f6638ds} for 
					MSP430F5438A and MSP430F6638, respectively.
				\item \textbf{Boards:} \emph{MSP-EXP430F5438} vs. \emph{MSP-TS430PZ100USB}\\
					MSP-EXP430F5438 \cite{5438aboard} is a prototyping board designed to make use of 
					microcontrollers from the family of MSP430F5438 and stands out for featuring the following 
					sensors and input/output elements: USB, JTAG, 5-position joystick, 2 push-buttons, 
					dot-matrix LCD display, accelerometer... (full list of features in \cite{5438aboard}). 
					Furthermore, it also equips the \emph{CC2420}, 2.4 GHz 802.15.4 radio module, 
					which make	
					this board specially convenient for this project's requirements. This module is directly 
					connected to the microcontroller allowing their communication through SPI. In addition, its
					USB connectivity provides a very useful serial output for debugging.\\
					
					% Nonetheless, despite the wide possibilities the previous board offers, the chosen one is
					% the MSP-TS430PZ100USB \cite{6638board}, which is quite simpler, due to its support for the
					% USB-capable MSP430F6638 microcontroller.
					Nonetheless, despite the wide possibilities the previous board offers, the MSP-TS430PZ100USB is selected. This board lacks serial port output and CC2420 compatible sockets, so the connection is to be done by hand and nor any of the other aforementioned artifacts available in the MSP-EXP430F5438 is present. But it provides the key feature required by the project: the ability to communicate the microprocessor with the board's USB connector.
			\end{itemize}
			
		\subsection{Shimmer}
		%Posibilidad de tras la epxlicación inicial meter dos secciones más que expliquen las diferencias entre los dos chips y las dos placas usadas(hablarlo todos)
		\subsection{802.15.4}
		\label{ssec:802.15.4}
		\subsection{FreeRTOS}
		\label{ssec:FreeRTOS}
		\todo{Pedir información y referencias a joaquin}
		\subsection{USB device \& USB host}

	\section{Description}

	%Esta parte es el cuerpo de la parte de investigación del proyecto, no se sabía si se podía, no se sabía cómo hacerlo, …

	% The hardware is the center section of the research in the whole project, not because there wasn't more research, but because nobody has researched this areas. At the begining we just know what we want to do but we have absolutely nor idea or clues about how we can do a very important number of our *milestones. In any cases we don't even know if our goals could be achieved, in particular a very important one, we need to connect a MSP430 to a Android where Android acts as host, and nobody achieve this before and therefore there are no information about that in forums or TI official support.\\

	The hardware related investigation is the main section of the research part of the project. Not that there wasn't any research involved in other areas, but some critical elements of this part had never been researched before. 
	At the beginning of the project specific objectives were established and main milestones elected, but absolutely no clue or direction for most of those milestones was available.
	Moreover, in some cases the feasibility of the proposed goals was unknown. Specifically the achievement of the very important objective of USB communication between an [TI's] MSP430 %(ponemos algo más de info o ya se sabe el modelo y todo?) 
	and an Android powered device assuming the latter the role of master and acting the former as a slave was something not done before and therefore no information was available  even in the Texas Instrument support site.\\

	%Plantea un reto porque toca todos los niveles, desde diseño de PCB a nivel componente hasta desarrollo a nivel de SO (kernel? <= investigar si kernel o SO)

	% The project suppose a chalenge because it involves every hardware level, form the lower levels as the PCB design of a device to higher levels as develop parts of a SO. *Aqui molaria poner algo más que dos tristes lineas. \\

	This whole development and research poses some quite interesting challenges as it involves working nearly at every abstraction layer present on device development, ranging from schematic capture and PCB design to operating system related development. The main issues to be resolved are:
	% La idea es refactorizar un poco lo que había en la scrap zone, introducir los puntos que debe resolver
	% el accesorio y después dar paso a los hitos, donde se relaciona cada uno con estos objetivos.
	\begin{itemize}
		\item \textbf{802.15.4 radio reception:}
			data packets are emitted from the monitoring Shimmer\texttrademark and are to be received and dealt
			by the accessory. In order to do so, FreeRTOS has to be equipped with a working implementation of
			the radio stack, which involves implementing part of ZigBee's MAC layer, described by IEEE 802.15.4, 
			as it is the underlying standard which the emitter nodes are based on. 
		\item \textbf{MSP430 USB handling:}
			just as the radio stack, FreeRTOS is not prepared to make use of the USB interface with which the MSP430
			board --namely TS430PZ100USB-- is equipped. Thus, it is necessary to add the essential code in order to 
			incorporate this functionality to the OS. This issue, in the same way as the previous one, is a critical 
			part for the system to work properly since it may likely be a source of errors unless it is perfectly 
			made --for instance, it could add transmission lags, packet corruption or loss--. 
		\item \textbf{MSP430-Android communication:}
			this issue is particularly relevant as no project has ever claimed to feature this kind of
			connection before. Hence, it means an additional challenge to be overcome since it will involve
			the modification or complete development of an specific driver for the MSP430-equipped device.
			This challenge is imposed by choosing the USB host alternative so that the Android device powers
			the accessory. On the other hand, USB device may ease this issue due to the already resolved 
			communication system with accessories made by Google, yet it implies the accessory will need
			an additional power source.
			Fortunately, it finally turned out not to be so tough as it was expected and the goals related
			to this issue were successfully fulfilled, as it can be read at \autoref{ssec:Android.USB}, 
			\nameref{ssec:Android.USB}.
	\end{itemize}
	
	% Let us present the following section...
	Over the following lines within the next section, the aforementioned challenges are detailed
	in the context of their own development stages. In addition, objectives and results of each one of
	these stages are also presented.
	
	\section{Milestones}	
	Due to the fair amount of time a essentially researching work like this hardware development will
	require, the foreseen schedule is subject to changes which may completely alter it regardless of 
	how irrelevant they may seem, and thus avoid the success of the project.\\
	
	Keeping this risk in mind, the hardware development was planned as a sequence of milestones,
	ordered by increasing complexity. Every milestone introduces a new technology, each one of them
	capable of covering the needs of the project in a more complete way. In other words, the previously
	presented issues are to be resolved as these stages are overcome.\\
	
	The schedule, then, starts with the usage of the Google Open Accessory Development Kit, based on Arduino,
	in order to make a prototype as first approach and, in this way, be able to parallelize software and
	hardware development --notice that both of them depend on each other--. This choice is made because of
	the well-known soft learning curve Arduino presents as well as the already prepared connection
	the Google ADK provides between accessories and Android devices.\\
	
	The next step in the development, once the application has been proven to work with a data-providing
	accessory, consists of making Android capable of detecting and communicate with an MSP430-equipped
	device. This stage can be qualified as very critical, since almost every part of the development
	depends on it.\\
	
	The following milestone arises as a consequence of the FreeRTOS port which was specified before: its
	target device, namely MSP430F5438A, does not support USB communication, and by extension the port itself.
	Thus, the work for this stage consists of providing FreeRTOS with USB support, which is to be employed
	with the new target device, MSP430F6638, which is member of the MSP430F66xx with highest performance,
	along with USB support.\\
	
	Next, the planning set the development of the communication between the emitter nodes and the device.
	The previously existing FreeRTOS port already supports the IEEE 802.15.4 standard MAC layer, so the
	work in this milestone involves needed modifications in order to get the port work properly with
	the new platform MSP430F6638. Although the first versions of the schedule considered this next step 
	at the same time of the previous one, in later revisions it was decided to keep them separated for 
	the sake of stability, derived from isolating both developments.\\
	
	Due to the division between the two previous stages, both developments, USB and radio stacks, are
	to be done separately. In order to resolve the likely communication problems this decision can cause,
	the next iteration the planning considers is reserved for validating and fixing their coexistence
	may provoke.\\
	
	Finally, the schedule adds two extra stages: the first, which may be dropped due to eventualities,
	considers the design of a miniaturized version of the receiver device so as to dispense with the
	TS430PZ100USB board, which is rather big and little portable. The second one is essential as it
	is reserved for possible needed fixings and refinements.\\
	
	All these stages are now detailed over the following subsections.\\

	\subsection{Arduino for Android USB Device Comunication}
	\label{ssec:Arduino.USB}
		The objectives for this first milestone objectives are:
		\begin{itemize}
			\item Acquire a suitable Android device prototyping environment, 
			\item manage correct communication between Android and a prototype device, and
			\item develop an application emulating desired behaviour.
		\end{itemize}
			
		% Milestone justification
		This milestone is set upon the consideration that it is a good approach in order to familiarize
		the team with communications between Android and USB devices. It also comes motivated by the
		previously supposed effort the connection between Android and a MSP430-equipped device involves,
		since it has not been done before as well as Android USB host mode support has been out for only
		two months.\\

		The process involved in the procurement of each objective is exposed next.

		\begin{enumerate}
			\item \emph{Acquire a suitable Android device prototyping environment}\\
				The first step in order to develop this USB device consists of looking for a platform which
				is able to provide the needed components and libraries so as to make the connection with
				Android feasible. Although several devices are available to serve as Android accessories (as
				it is explained at \autoref{ssec:Arduino}),	Google ADK is the chose one since it is based on
				Arduino: a considerable amount of libraries and documentation is to be expected. Particularly,
				USB host mode library is the main reason for choosing it.

			\item \emph{Manage correct communication between Android and a prototype device}\\
				Once the target platform is chosen, the next step to be done consists of developing some
				test firmware for the Arduino as well as modifying consistently the Android application
				so that a proper communication between both devices can be established and proven. A period 
				of learning and adaptation is required for the team to achieve this.

			\item \emph{Develop an application emulating desired behaviour}\\
				For this milestone to be fulfilled, it is required that the firmware developed for the
				Arduino can emulate the behaviour of an actual sample-receiving device. Thus, the
				microcontroller board is programmed in order that it sends formatted data as the Android
				application is expecting to receive. In this way, it is visually verifiable that the
				data transmission is performing properly.
		\end{enumerate}

		Although Google refers its ADK as an accessory developing platform, it is noticeable that it is
		in fact used as a prototyping tool, and hence this development is not to be employed as a part
		of the final state of the system. As opposed to this, this stage of the project is considered as
		training transition to more technical developments to be engaged later.

	\subsection{MSP430 for Android USB Host Comunication}
	\label{ssec:Android.USB}
		Objectives for this stage are:
		\begin{itemize}
			\item Supply MSP430 with USB protocol application functionality,
			\item research Android USB protocol related functionality,
			\item make Android OS recognize MSP430 when plugged, and
			\item manage communication between Android and MSP430 via USB.
		\end{itemize}

		\begin{comment}
		%Plantear de manera lo más cronológica posible remarcando que aui estabamos totalmente a ciegas, no sabíamos como ibamos a alcanzar nuestro objetivo, solo teníamos una API USB proporcionada por TI para comunicación USB con windows: 
		Unlike the arduino's part where we know what we have to use and how we wil use it, now we found 
		that we just have an Texas Instruments(TI) API to comunicate the MSP430 with windows and our goal. 
		Both MSP430 microcontroller and MSP430 board was new devices because all the existing ones have no 
		USB port, the microcontroller is a MSP430 6638 and the board is a TS430PZ100USB(link to the MSP430 section).\\
		\end{comment}
		
		Due to the need of USB support, both the microcontroller and its board models are required to be
		as new as possible, since older models lack this feature. Therefore, the MSP430F6638 microcontroller
		as well as the MSP-TS430PZ100USB board are selected for this development, as it is explained at
		\autoref{ssec:msp430}, \nameref{ssec:msp430}.\\
		
		Prototyping with Arduino during the previous milestone constitutes the preparation for the development
		to be done in the present one. Yet, whereas Android and Google ADK are designed to interact with
		one another, Android is not prepared for supporting a device like a MSP430 out of the box. So, in order to
		communicate both devices, an API from Texas Instruments \cite{TIUSB} is the only available tool, though
		it is designed for Windows OS.\\
		
		\begin{comment}
		%Introduccion y no va (android usb protocol)
		Initialy in order to test the TI API we check some of the multiple examples, in our case was the 
		CDC examples that be usefull to learn about basic concepts of what we try to do. Also we try to 
		read part of the generic(not dependant of the MSP430 device nor USB protocol)documentation that 
		TI provides. And it didn't start too good, the first test with a example provide in the API didn't 
		work in windows, its target SO.
		\end{comment}
		
		Among all the contents the TI API includes, its examples result particularly useful for the project
		requirements. Concretely, examples about CDC (Communications Device Class) protocol set the needed
		basis to fulfill the objectives for this milestone. Nonetheless, several examples does not seem to
		function, not even in Windows, its target operating system. Thus, the next step consists of studying
		both the examples and documentation from TI so as to fulfill the objectives.
		
		\begin{enumerate}
			\item \emph{Supply MSP430 with USB protocol application functionality}\\
				\begin{comment}
				% MSP (1)
				This API contains any simple aplications for windows, to comunicate by certain protocols with 
				external devices or enumeration tools; an extense documentation about the API and about the USB 
				characteristics of its devices; and a huge suite of examples that implements a lot of USB protocols 
				to comunicate the MSP430 whith windows such as Communications Device Class (CDC), Personal Healthcare 
				Device Class (PHDC), Human Interface Device (HID) in traditional and datatype implementations, Mass 
				Storage Class (MSC) and combinations of any of them. At this point we have an inmense amount of information 
				and we don't really know what can be usefull for us.\\
				
				%LE FALTABA UN PUTO RELOJ DE 4MHZ, MSP (2)
				After a good time of investigation we found that it can't initialize a certain clock, this clock 
				don't seems to be in the board and we there are not much references about it. Finally we found a 
				little paragraph in one API document that metion the chance of a USB needed clock was not included 
				in any kind of boards and a recomendation of how should be this clock, when we buy a clock of this 
				features we can check that this was the problems, the TI example finally works.

				%En ubuntu no va, mal rollito, MSP (3)
				All this tetst was carried out in a virtual machine with windows usually running over other windows, 
				but in a casual situation when this virtual machine was running over ubuntu we discover that this examples 
				not only don't work in ubuntu, somthing that we assume, but in addition ubuntu can't even detect our MSP430, 
				this fact worry us because android is an UNIX based SO just like ubuntu. \\
				\end{comment}
				
				As it was previously said, TI API contains a huge amount of information in the shape of:
				\begin{itemize}
					\item \textbf{Extense documentation:} about the API itself and features of TI devices. 
					\item \textbf{Sample Windows Applications:} made to communicate Windows PCs with external devices or
						enumeration tools.
					\item \textbf{Suite of examples:} each of them implements a combination or one of the following
						USB protocols: Communications Device Class (CDC), Human Interface Device (HID) or Mass Storage Class (MSC).
				\end{itemize}
				
				The same problems, though, keep these examples from working with the target MSP430 device. As a result of
				an intense investigation the trouble is found to lie within one of the device's clocks which, according to 
				the API documentation, may need an specific tuning --particularly, 4MHz--. Moreover, in spite of the fact 
				that the board seems functional, it actually lacks that certain clock. As expected, TI example properly
				works with the addition of the 4MHz clock.\\
				
				In addition to the previous hitch, another one arises upon changing the current workstation. Every work
				during this development is carried out within a Windows virtual machine running over Windows, yet
				the examples stops working when the host machine is changed for another one running GNU/Linux, 
				particularly Ubuntu. Despite having achieved the present objective, this fact reduces the expectations
				for the next one since Android, the target OS, is also based on Linux kernel.
				
			\item \emph{Research Android USB protocol related functionality}\\
				\begin{comment}
				% USB Android
				%Líneas de investigación
				When we try to connect our MSP430 running one of this examples mentioned before we obvserve that, 
				as we expect, android also can't detect it. Initialy we think our only posible solution to this 
				important trouble was make or found a unix driver, then we investigate that way. \\

				%Driver propio para linux
				%Driver propio para android (ambos sobre driver TI MSP430 para MAC)
				%[Mencionar Riesgos asumidos al abandonar línea driver propio y seguir investigando]
				After much searching we see that there are no drivers for unix made by anyone, the closest driver 
				we find was a MAC driver(that we are not sure that it going to work) and assuming that we will 
				need to do a driver based on it, we consult some cualified personal in this area that said us 
				to forget the idea of a generic UNIX driver and focus in an android driver, but also recommend 
				that keep researching other ways to communicate android and MSP in order to avoid the develop a 
				driver for MSP430 wich can be a very difficult work. This advice and the fact that there are no 
				warraties of the develop of a driver was succesfull, we decide to leave the driver idea and follow 
				investigating. That was a very risky decision because we have spended a lot of time in this way and 
				we don't know if we will find other idea to ahieve this hard milestone.\\

				%Encontramos mención a driver genérico HID en android + TI’s HID api for MSP430s
				After a few days of unsuccesfull days of investigation, we find in a forum a small mention in a 
				comment about android actually implements HID protocol. This protocol was also suported by the 
				MSP430 API, and although the information was found in a not very condiable place, we find it 
				enought to put the full team to work in this, ones made the android USB host(link) and others 
				find a HID aplication into the API to load into the MSP430. The second objetive was attenpted 
				first, with this we can check that our android device finally detect our MSP430. \\ % Lo conectas y va
				\end{comment}
				
				As it is expected because of its performance with Ubuntu, Android is not able to recognize the
				target device, and hence the example cannot be tested. In order to solve this inconvenience, two
				solutions are considered: search for the proper driver, if any, modify an existing one or develop
				it from scratch.\\
				
				Nevertheless, the search concludes with no results, so it is assumed that this particular driver does
				not exist. The most similar resource found consists of a driver for MAC OS X, which would have to 
				be consistently modified in order to make it work. However, this idea is dropped due to the advice
				obtained from a qualified source that suggests exploring other methods to make the communication
				feasible apart from developing a driver, which may be too harsh and spend too much time.
				A quite considerable risk is assumed by doing so as there are no warranties of finding a proper solution.\\
				
				Fortunately, further research drives to a successful alternative: it is found that Android actually
				implements HID protocol, which is also supported by MSP430 through the Texas Instruments API. Finally,
				upon loading the proper HID application into the MSP430 from the TI API this objective reaches its fulfillment.
			\item \emph{Make Android OS recognize MSP430 when plugged}\\
				Once both two first objectives of this milestone are achieved, MSP430 recognition by Android is
				inmediate when the connection is set since all the needed work is already done.
			\item \emph{Manage communication between Android and MSP430 via USB}
				\begin{comment}
				%Probamos y funcionó (más trabajo software, enlazar a capítulo) , comentar la importancia del hardware en esta parte de la programación android debido a que ese código es muy dependiente de como implemente el protocolo HID(que proporciona mucha libertad)el MSP lo que supuso mucha investigación y continuo acceso a manuales tanto de la API como del propio MSP(ese comentario iria aqui o en la parte de software?Hablarlo todos)
				That great news helps the android development team, than in this moment becomes the full team, 
				to succesfully imlements the android USB host communication in our aplication in just a weekend. 
				Android USB host comunication was higly dependant of what device was in the other side of the 
				communication, thus everybody was needed in this hard and delicated part of the develop to 
				investigate the high cuantity of manuals contained in the API in orther to find and implement 
				all this particularities. Finally we luckily discover that now, our android and MSP430 can 
				also comunicate each other trought our aplication. \\
				\end{comment}
				At this point, the work still to be done consist of modifying the Android application in order
				that it can read data from the receiver device. However, this is not a trivial work as Android
				USB host communication is highly dependant on the device which it has to read from. Thorough
				investigation over the API manuals is required to discover every detail and make it work.
		\end{enumerate}

		
		%Conclusion
		\todo{de-draft this conclusion}
		That was wtih no doubts the harder and more dangerous part of the research, there was a lot of 
		chances to do not achieve our goal and the hard work of all the team was essential. This milestone 
		suppose a very important fact not only in hardware part but in all the project because now, we can 
		more accurately schedule all of the project. \\


		\subsection{USB in FreeRTOS}
		\label{ssec:USB.FreeRTOS}	
		This milestone's objectives are:
		\begin{itemize}
		\item Validate the use of FreeRTOS in the new tarject MSP430
		\item validate USB API utilization viability in conjunction with FreeRTOS in MSP430,
		\item correctly integrate USB API into FreeRTOS, and
		\item manage USB data sending in FreeRTOS.
		% REALIZACION: port testing application to FreeRTOS task system.
		\end{itemize}
		\todo{Antes de cambios o mierdas, miniintroduccion del freeRTOS y referencía a su capitulo?}
		

		\begin{enumerate}
		\item \emph{Validate the use of FreeRTOS in the new tarject MSP430}\\
		%Pequeños cambios en FreeRTOS para soportar la nueva plataforma 6638
			Before to start the real objetive of this milestone we need to adapt the FreeRTOS main functionalities to the new MSP430, that been a new device was not actualy supported by. This mind the creation of a good number of new clases, most of them was excatly equal to their homonimes for the MSP430 5438A but other needs some little modifications.\\

		\item\emph{Validate USB API utilization viability in conjunction with FreeRTOS in MSP430}
		%Adaptación de la API para hacerla funcionar en un SO basado en tareas como es el FreeRTOS
		%Importantes riesgos de conflictos a nivel de compartición de recursos hardware, que con cuidado(más con suerte que con cuidado) pudieron ser esquivados.
			The next need in the milestone was port as soon as posible the TI USB API to a task-based SO like FreeRTOS whithout taking too much care about its correction. The introduction into the FreeRTOS was prety problematic because the size of just the the API was near to the size of the FreeRTOS. Plus, there are a very important risk, USB uses a important number of resources as pines or clock that can be also used by the FreeRTOS to another task, specially risky was the clock because both need a clock, but the selected board have 2 clock spots, and taking care in the port all this themes could resolved and everything works fine.\\

		\item\emph{Correctly integrate USB API into FreeRTOS}
		%Estaba hecho pero mal, había inclusiones puestas donde no se debía y ya no se podía ejecutar ese mismo código en el shimmer o el antiguo MSP430.
		Once it's done our preocupation was how to order all this files in a correct way, because our free RTOS was a multiplatform SO that must work in MSP430 5438A, Shimmer, and now, the new MSP430 6638 where take place this develope. In the first port where the multiplatform ability of FreeRTOS was not a problem we lost this funcionality. This separation of tasks help pretty much in focus the first steps in this milestone. This new functionality have to be included just in the supported platform, the MSP430 6638, without affect the other ones as before. Using the preciding generalization needed to cohexists Shimmer and MSP430 5438A as a guide this port was not too traumatic.

		\item\emph{Manage USB data sending in FreeRTOS}
		%Aplicacion de prueba que manda el abecedario y en el otro lado vemos que llega perfectamente
		Finally with the USB API correctly integrate and its functionality validated, the implementacion of a closer program to the expected funcitionality of the device was attempted, sending a already know character secuence of 1 byte long, and checking
that it's been recived in the other side. This test was satisfactory and the milestone was finally achieved.


		\end{enumerate}
		\subsection{802.15.4 in FreeRTOS}
		\label{ssec:802.15.4.FreeRTOS}	
		This milestone's objectives are:
		\begin{itemize}
		\item Validate current implementation of 802.15.4 in FreeRTOS in old MSP430,
		\item manage connection to the CC2420 radio module to target MSP430 device,
		\item port implementation of 802.15.4 to target MSP430 device, and
		\item prepare such implementation for actual usage.
		\end{itemize}

		%Estado de la capa MAC
		At the beginining of this milestone there are a port of the needed part of the MAC layer of the 802.15.4, that have been tested in just certain conditions, like send of medium lenght packets.\\

		\begin{enumerate}
		\item \emph{Validate current implementation of 802.15.4 in FreeRTOS in testing MSP430}\\
		%Radio probada en una placa que no proveía USB pero sí salida por puerto serie para simplificar trabajo y facilitar la depuración,		
		As we are not sure of the right working of MAC layer in our system we decide to test it with the old board and microchip that provides serial port output that is extreamly usefull in the debug of a real-time system like this. This result to not works for a certain problems as although it's able to recibe 802.15.4 packets it's not programmed to it, and the max size packets wasn't recived correctly.\\

		%se llevó luego a otra con USB pero sin soporte ni software ni hardware para la radio, obligando a mapeo manual de pines,
		\item\emph{Manage connection to the CC2420 radio module to target MSP430 device,}
		Once the right working of the MAC layer is tested, it's time to port it to the new board and microchip, that implies a lot of troubles because the TS430PZ100USB have no conection to a CC2420 radio module. This mean that a full study of the 100 aviable pins to discover which ones are actually unused by both the SO and USB comunication. The radio modlule alse need a particular kind of pins in some cases that there are no very abundant. With this study and the mapping of pins made\todo{(mencionamos a carlos aqui?)} board and radio module is sent to be weld.\\

		\item\emph{Port implementation of 802.15.4 to target MSP430 device}
		%que ahora si con más mañana que suerte se hizo bien y no dio conflicto como veremos ahora aunque si hubo que modificar algo más de código dado que la nueva placa necesitaba iniciar los pines de la radio de otra forma.
		While the board is available, we addressed the programming the pin mapping for de MSP430F6638 into it's class in the FreeRTOS using as base the MSP430F5438A pin mapping class. This is a particulary delicated code and was carefully developed, because if just one thing is not perfect the radio simply didn't work at all and the potencial error will be hard to discover. \\

		\item\emph{Prepare such implementation for actual usage.}
		% No va, había que cambiar un par de cosas en la inicialización, lo descubrimos en un ejemplo de TI
		With both, board and codding finished the radio was tested and it didn't even trun on. The answer to this trouble was found in a code examples provided by TI for the MSP430F6638, specifically in the \textit{Universal Asyncronous Reciver/Transmiter(USART) initialization code} that reslut to differ slightly of the old MSP430 USART initialization.\\

		%Conclusion
		%Pequeños cambios para adaptarse a nuestras actuales necesidades y conclusión
		Finally some small changes was done in order to adapt it to our project needs. With this a fully funcitonal MAC layer working on the MSP430F6638 was achived and just the potential coexistence with the USB was on the air.\\

		\todo{Comentar que este trabajo se hizo en colaboración con joaquin?}
		\end{enumerate}


		\subsection{802.15.4 \& USB coexistence under FreeRTOS}
		\label{ssec:802.15.4.USB.FreeRTOS}	
		This milestone's objectives are:
		\begin{itemize}
		\item Assess conflict-free coexistence of current implementation of both USB and 802.15.4 modules in MSP430, and
		\item manage sending data received from 802.15.4 via USB.
		\end{itemize}
		%Intro
		With both USB and 802.15.4 communication working separately we need to test that they can work together. There are 2 main risks; hardware, because any pines used in 802.15.4 can be used also in USB and software, because the time between a radio interruption and the next radio interruption could be too short to send the data trought USB.\\

		\begin{enumerate}
		\item\emph{Assess conflict-free coexistence of current implementation of both USB and 802.15.4 modules in MSP430}
		%Riesgos de conflicto hardware entre Radio y USB
		The hardware risk was adviced much before the begining of this milestone, and when we made de pin mapping we keep in mind this risk, thanks that, this risk was avoided.\\

		%Riesgos de conflicto software entre Radio y USB sobre FreeRTOs
			%(Falta de tiempo para enviar y recibir)
		%Darle mucho peso a los riesgos y ver como tratar el tema de que no hubo prácticamente ningún problema con ellos(solo al de tiempo para enviar y recibir) sin que se note que este punto no tuvo mucho peso
		\item\emph{manage sending data received from 802.15.4 via USB.}
		After develop an aplication to send data recived trought USB is noticed that, however the software risk was initialy not avoided because the Shimmer send data paquets too fast and MSP430 can't manage this amount of information to send it trought USB and some packets was lost. A little adjusts was necesary, the packets sent was concentred in the start of the available time slots then, we spaced it, sending the same number of packets but with the same time between packet and packet, filling the whole aviable time slots. \\
		\end{enumerate}
	
		%Conclusion
		%Este hito hacia falta aunque fuese breve para tener el sistema lo más estable posible antes de lanzarnos a probar la aplicación final
		Now, our system was finally able to send trought USB the packets recived in radio with no losses. This achievment was very important before develop and test the final aplication with several real-time restrictions.\\

		\begin{comment}
		\subsection{Final aplication develop}
		\begin{itemize}
		\item Obtention of a shimmer final aplication,
		\item Obtention of a Android aplication, and
		\item Test the whole system in its real use.
		\end{itemize}
		
		%Le pedimos a fran(mencionamos a fran?) que nos pasara una aplicación que enviara todo lo que era capaz de enviar el shimmer por radio, no la tenía y la tuvo que hacer, acabamos la parte sofware para parsear los datos llegados por USB y a probar.
	
		%Sabiamos que funcionaba con nuestros ejemplos pero faltaba comprobar si funcionaría con la aplicación objetivo del shimmer que tenía unas restricciones de envío más altas y como no podía ser de otra forma no funcionó. Nos agobiamos, rafa se temió lo peor, el shimmer enviaba mal, el tablet parseaba mal, pero nada de esto resolvía los problemas y al final tuvimos suerte y se arregló.
		\end{comment}

		\subsection{MSP430 based device design}
		\label{ssec:device.design}	
		\begin{itemize}
		\item Board exhaustive analysis,
		\item Capture of the schematic, and
		\item PCB design and route.
		\end{itemize}
		%Algo en plan, finalizada la investigación y gracias a que pusimos mucho empeño en acabar con tiempo suficiente fuimos capaces(como de tener la capacidad, no de conseguir(be able to)) de diseñar un dispositivo a partir de los componentes presentes en la placa de prototipado.
		%Identificación de componentes de placa 
			%(pines de expansión)
			%(cambio de componentes por otros)
			%(eliminacion JTAG y reutilizacion Radio)
		%Captura de esquemático
		%Diseño PCB
		%Rutado PCB

		\subsection{Final Validation and Release Candidate Version Developmentn}
		\label{ssec:Final.Validation}	
		\begin{itemize}
		\item Final validation of final aplications with prototyping hardware
		\item Final validation of final aplications with final hardware
		\end{itemize}
		
		%Esteeee si, mencinamos que esta milestone existía porque había que hacer las pruebas pertinentes ahora que todo estaba completo y comentamos lo que sea que vaya a pasar cuando esté todo.

		\section{Final Product}
		%Descripción de que es lo que hemos conseguido(es posible que con la evolución alguien que se lea las iteraciones no consiga una clara visión de cual es el estado final del dispositivo, por lo que esta parte me parece muy importante)
		
		Primer parrafo, finalmente, despues de toda esta investigación podemos afirmar que es lo que tenemos, un dispositivo totalmente diseñado por nosotros, de un tamaño reducido y comodo de usar, y que es capaz de comunicarse tanto con un android(a partir de la 3.1)(moviles y tablets) como con widows y es capaz de recibir paquetes por la radio siguiendo el estandar 802.15.4. 

		De hecho el hardware que lleva lo hace potencialmente usable para cualquiera que requiera un modulo USB que reciba señales de radio por 802.15.4 y las mande por USB por lo que su potencial es mucho mayor que el que se le da en el ambito concreto del proyecto y podría tener muchos más usos que el dado.

		En el ambito concreto del proyecto, el producto nos sirve como medio para que el android pueda recibir la onda ECG capturada y enviada por el shimmer, es interesante por tanto que el tamaño del dispositivo diseñado es realmente pequeño por lo que será comodo para llevar para las personas que por tener un problema necesiten monitorizarse, además gracias a su bajo consumo la vida util de nuestro dispositivo android no se verá muy reducida con su uso, y más importante aún  gracias a esto el shimmer no necesita el bluetooth y puede enviar las señales por radio de manera que consume mucho menos y ve alargado su tiempo entre recargas de 5 o 6 horas a 5 o 6 dias. Especialmente interesante de este dato es el hecho de que con la radio el shimmer supera el umbral del día de funcionamiento por lo que cargando el dispositivo por la noche(sin necesidad de desconectarlo de los sensores, para no interrumpir la monitorización) el paciente podría estar monitorizado indefinidamente, en nuestro caso nuestro paciente podría incluso irse de fin de semana sin preocuparse de llevar el cargador de la batería.

		Fuera del ambito del proycto, este dispositivo podría convertirse en un estandar ya que permite recibir cualquier onda 802.15.4 y pasarsela tanto a android como a windows, además con unas sencillisimas modificaciones en el código el dispositivo puede asumir el rol de ser él el que manda los datos que recibe por USB por lo que disponiendo de dos de ellos, podríamos comunicar dos aparatos cualquiera(android o windows) por 802.15.4, como por ejemplo hacer unos walkies de bajo consumo para android o..(aqui más chorradas)

		Finalizar con las conclusiones sobre el desarrollo:

		La investigacion es muchas veces imprevisible y lo que llevas semanas sin conseguir progresos queda resuelto en un buen día, por lo que los desarrollos en este campo no tienen sentido sin planificaciones flexibles como la que planteamos. 

		Trabajo futuro en esto, sería como mucho hacer el código que he comentado antes que haga el paso inverso y reciba por USB y mande por radio pero si de verdad interesa se puede hacer antes de la presentación, pero poco más

		Como dificultades cabría mencionar que la documentación de TI no es todo lo clara que merece su tamaño, por lo que a veces resultaba complicado, o inlcuso podemos decir que una dificultad era soldar piezas diminutas a la hora de hacer el prototipo pero que eso se resolvió gracias a que lo hizo carlos.
















