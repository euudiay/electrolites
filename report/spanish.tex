\chapter{Resumen en español}
\label{ch:resumen}

	En este proyecto se expone la investigación realizada con el objetivo de desarrollar un receptor del estándar de redes de área personal inalámbricas 802.15.4 del IEEE para dispositivos Android a través de USB, aplicado a un sistema completo de monitorización electrocardiográfica (ECG), así como el proceso de desarollo del mismo. Este sistema se enmarca en el ámbito de la atención sanitaria personal: su principal aplicación es la montorización del estado del corazón por parte de un particular, eliminando la dependencia, respecto a esta tarea específica, con los sistemas de atención sanitaria tradicionales.\\

	Recientemente ha surgido un gran interés, tanto en el ámbito académico como en el industrial, en la producción de sistemas de monitorización ECG portátiles y de bajo consumo, llegando a ser una de las principales aplicaciones de las redes de sensores corporales inalámbricas.\\

	Para maximizar la portabilidad, en el desarrollo de estos sistemas se han empezado a emplear dispositivos móviles de gran capacidad de cómputo, particularmente smartphones, debido a la gran difusion que han tenido en los últimos años. En el 2011  se presentó un sistema, colaboración entre la Universidad Complutense de Madrid (UCM) y la École Polytechnique Fédérale de Lausanne (EPFL), para monitorización ECG de ámbito personal empleando un iPhone como visualizador y Bluetooth como tecnología de comunicación inalámbrica\todo{Citas en el resumen?}.\\

	De los resultados obtenidos por ese proyecto surge la presente iniciativa que trata de llevar al siguiente nivel las características inherentemente buenas de bajo consumo y bajo coste del mismo mediante la aplicación del protocolo 802.15.4, de mucho menor consumo energético que la tecnología Bluetooth, y la sustitución del dispositivo iOS por uno basado en Android, debido a su mayor accesibilidad y el menor coste, en general, de éstos.\\

	El sistema desarrollado en este proyecto presenta al usuario una representación visual de su onda ECG en tiempo real, resaltando puntos relevantes para simplificar la compresión de los datos mostrados. También muestra información sobre el ritmo cardiaco, y toda esta información es almacenada de forma transparente al usuario para su posterior consulta.\\

	Esta funcionalidad es posible gracias a la operación conjunta de los tres dispositivos que forman el sistema de monitorización: el nodo de delineación ECG, el receptor 802.15.4 y el dipositivo Android que actúa de interfaz con el sistema. El nodo de delineación ECG va conectado a la red de sensores corporal del usuario y se encarga de la captura y posterior análisis de la onda ECG, así como de la codificación y envío de la misma de forma inalámbrica. El receptor 802.15.4 conectado al sistema Android a través de USB controla la recepción de datos a traves de dicho protocolo y el envío de la información recibida al dispositivo Android. Éste actúa como decodificador y visualizador en tiempo real, y, como interfaz con el sistema, gestiona las conexiones inalámbricas y almacena y muestra los datos recibidos.\\

	Los objetivos del proyecto son, entonces, el desarrollo de la aplicación para dispositivos Android, la producción del receptor 802.15.4 y la comunicación de ambos con un nodo delineador ECG ya existente.\\

	La aplicación para dispositivos Android, como ya se ha mencionado, es la interfaz con la que el usuario interactúa con el sistema. Su diseño sigue las prácticas comunes de aplicaciones para este sistema operativo ya que el objetivo es que la curva de aprendizaje sea, si no nula, muy suave. Los motivos para emplear Android como sistema operativo base son tres: la importancia de éste entre los sistemas operativos móviles, el mayor rango de precios de los terminales que lo soportan, que permite una mayor difusión del sistema debido a la existencia de dispositivos de precio más reducido, y la naturaleza libre y de código abierto del entorno de desarrollo, característica que facilita la posterior expansión del sistema. Todos estos motivos pueden resumirse en que el empleo de Android como sistema operativo permite el acceso de un mayor número de usuarios al mismo.\\

	En cuanto al nodo delineador de la onda ECG, el objetivo es emplear uno ya desarrollado para el proyecto. Esto es así porque tanto el nodo delineador como la red de sensores corporales que capturan la onda ECG son sistemas especialmente complejos cuyos desarrollos ocuparían, cada uno, un proyecto de la envergadura del actual.\\

	El nodo delineador que se emplea en el proyecto es obtenido en el proyecto de la UCM y la EPFL antes mencionado. Este nodo se desarrolló inicialmente como un delineador de electrocardiografía en tiempo real de bajo consumo con capacidad para envíar los datos de forma inalámbrica a través de Bluetooth. En otro esfuerzo conjunto entre la EPFL y la UCM se dotó al nodo de funcionalidad para el envío empleando 802.15.4, pero tan sólo al nivel necesario para realizar algunas estimaciones de consumo. Incluso con la utlización real del estándar 802.15.4 estando aún por llegar, el nodo presenta las características necesarias para convertirse en un excelente punto de partida para el alcance de los objetivos del proyecto.\\

	El componente más importante del sistema en cuanto a este proyecto se refiere es el receptor USB de 802.15.4 en tiempo real para dispositivos Android. Su desarrollo es una necesidad puesto que los dispositivos Android actualmente no proporcionan soporte para el protocolo 802.15.4, aunque sí para otros como Bluetooth o Wi-Fi. Considerando que en el nodo delineador que se toma como punto de partida para el sistema la operación que supone mayor consumo de batería es la utilización de la infraestructura Bluetooth, como confirman los estudios realizados por la UCM y la EPFL, la producción del receptor 802.15.4 es, pues, imperativa.\\

	La existencia de otros proyectos con el objetivo de la dotación a dispositivos móviles del estándar 802.15.4 es un hecho conocido desde el inicio del proyecto, aunque éstos no tengan como objetivo el campo de la biomedicina o la monitorización personal. El motivo para no apoyar o emplear alguno de ellos es doble: por un lado, al comienzo del proyecto éstas iniciativas se encuentran o bien inconclusas o bien paralizadas; más aún, todas son proyectos aislados, generalmente desarrollados por una única persona y sin ningún tipo de soporte oficial ni garantías de finalización. Por otra parte, y siguiendo la idea de obtención de un sistema de tamaño reducido, el objetivo de este proyecto es emplear el dispositivo Android como maestro en la comunicación USB, de forma que el receptor 802.15.4 obtenga la alimentación a través de él, eliminando así la necesidad de emplear una batería que incrementaría el tamaño del receptor. Ninguno de los proyectos existentes emplea la capacidad de los dispositivos Android para asumir el rol de maestro en la comunicación, así que no son de utilidad real para el proyecto.\\

	Por todo lo anterior, el desarrollo del proyecto viene motivado por dos razones principales: la 
	potencial utilidad que demuestra ser un sistema como este y el interés que suscita a nivel
	académico.\\

	Según la Organización Mundial de la Salud (OMS), las enfermedades cardiovasculares representan la
	principal causa de mortalidad a nivel mundial. Sin embargo, la vigilancia que requieren las
	personas afectadas no es asumible, por lo general, por los sistemas de sanidad tradicionales. La
	monitorización doméstica continua supondría una gran ayuda para estas personas, pero todavía
	supone un objetivo a cumplir.\\

	En un escenario como este, las redes inalámbricas de sensores corporales (wireless body sensor
	networks, WBSN) demuestran ser herramientas de monitorización eficientes y asumibles
	económicamente. Particularmente, las WBSN aplicadas a la monitorización electrocardiográfica son
	de gran utilidad en el seguimiento de enfermedades cardiovasculares. De hecho, proyectos como el
	de la UCM y la EPFL mecionado anteriormente promueven la expansión de sistemas como estos.\\

	Además, con el uso de dispositivos portátiles, en particular smartphones, para mostrar los
	resultados de la monitorización se persigue la expansión de estos sistemas en todos los ámbitos
	de la sociedad, ya que permiten integrar el sistema en dispositivos habituales, evitando así la
	necesidad de cargar con aparatos adicionales. La elección de Android como plataforma reponde a
	los factores que se mencionaron anteriormente, con el objetivo principal de favorecer la posible
	expansión del sistema.\\

	Asimismo, se busca también el uso más amplio de este sistema con el aumento de la duración de la
	batería y la reducción del tamaño de los nodos de monitorización: un consumo de batería menos
	exigente permite un número menor de interrupciones en la monitorización debidas a la necesidad de
	recargarla menos frecuentemente, mientras que el menor tamaño del dispositivo facilita la tarea
	de utilizarlo constantemente. La utilización del estándar 802.15.4 responde a la búsqueda de la
	realización de estos objetivos.\\

	En conjunto, el uso de un dispositivo Android con capacidad de recepción según el 802.15.4 en un
	sistema de monitorización electrocardiográfica continua rebaja los requisitos de consumo de batería
	y tamaño del sistema completo, y de esta forma mejora la posibilidad de ser transportado y relaja
	las restricciones en su aplicación.\\

	Desde un punto de vista académico, la principal motivación que plantea el desarrollo de este
	proyecto es el hecho de que reune prácticamente todas las áreas de la carrera de Ingeniería
	Informática. Desde el mismo comienzo el proyecto requiere de desarrollo tanto software como
	hardware, e implica la investigación en ámbitos externos al de la propia carrera, además de diseño
	y programación en varios niveles de abstracción distintos.\\

	Se espera también que las tecnologías basadas en la especificación del estándar 802.15.4 del IEEE
	como ZigBee verán incrementada su relevancia a medio plazo debido a sus potenciales aplicaciones en
	el campo de la domótica y otras áreas.\\

	Además, tanto el desarrollo de aplicaciones como de accesorios para smartphones u otro tipo de
	dispositivos portátiles son unas de las prácticas más habituales hoy en día en el ámbito del
	desarrollo de sistemas informáticos, y representan unas de las actividades más demandadas. Por
	tanto, entrar en contacto con ellas proporciona una experiencia adicional en un campo puntero como
	este, lo que amplía nuestro rango de habilidades y, por consiguiente, incrementa nuestro valor en 
	el mercado laboral.\\

