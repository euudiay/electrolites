\chapter{Abstract}
\label{cha:abstract}

	{\small 
	\paragraph{} People affected by specific cardiovascular diseases require of a constant monitoring of their vital signs, to which end specialized, high priced and big sized equipment is employed. Reduction of the energetic requirements and improvement of the portability are the objectives for the next generation of monitoring systems. This document presents the development of a low-power wireless ECG monitoring system for Android devices. By using the mobile phone or tablet of the user the total amount of needed devices is limited, and the application of the 802.15.4 wireless communication standard substantially decreases the energetic consumption when compared to wider spread ones like Bluetooth. The development of an USB 802.15.4 receiver device and the Android monitoring application results in a system targeting an operation as unintrusive as possible. Systems like this one have proved to be highly useful and a generalization of their employment is to be expected.
	\paragraph{Keywords}	
	ECG, Android, MSP430, FreeRTOS, Shimmer, USB, 802.15.4
	}\\
	
	{\small 
	\paragraph{} Las personas afectadas por ciertas enfermedades cardiovasculares requieren de una estrecha vigilancia de sus constantes vitales, lo cual supone el empleo de equipos especializados de elevado coste y tamaño. La reducción del consumo energético y el aumento de la portabilidad son los objetivos de la próxima generación de dispositivos de monitorización. En este documento se presenta el desarrollo de un sistema inalámbrico de monitorización electrocardiográfica portátil de bajo consumo para dispositivos Android. Al reutilizar el terminal del usuario se reduce el número de dispositivos necesarios, y la aplicación del estándar de comunicación inalámbrica 802.15.4 disminuye el consumo de energía de forma significativa respecto al uso de otras alternativas como Bluetooth, la más empleada en este ámbito. El desarrollo de un receptor USB 802.15.4 junto con la aplicación de monitorización para Android resulta en un sistema orientado a ser lo menos invasivo posible en la vida del usuario final. Sistemas de estas características han desmostrado ser de gran utilidad y se espera un uso generalizado de los mismos en casos de necesidad de monitorización constante.
	\paragraph{Palabras clave}
	ECG, Android, MSP430, FreeRTOS, Shimmer, USB, 802.15.4
	}



\newpage % disclaimer
	\paragraph{}
	Pablo Fernández, Rafael de la Hoz and Miguel Márquez authorize Complutense University 
	of Madrid to spread and use this report and the associated code, multimedia content and its results with academical and non-comercial purposes,
	provided that its authors shall be explicitly mentioned.
	\begin{center}
    	\thedate \\
	    \vspace{5.5in}
		Pablo Fernández\hspace{0.75in}
		Rafael de la Hoz\hspace{0.75in}
		Miguel Márquez
	\end{center}
