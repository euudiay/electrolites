\chapter{Results}
\label{cha:results}

	\section{Final state}
	\label{sec:end-state}
	
		% Describes the fulfillment of this project.

		\todo{Describe battery usage, size requirements actual data}
		% The production of a fully functional, low cost and low sized ECG monitoring system which employs an Android device as the user interface, and it's USB 802.15.4 receiver is completed.\\

		The production of the fully functional, low energy requiring, low sized ECG monitoring system employing an Android device as the user interface and 802.15.4 as the wireless data transferring protocol has been achieved. The system provides all the required functionality: real-time ECG wave data visualization both from 802.15.4 and Bluetooth nodes and storing of the received data in log files for afterwards visualization of these, as well as visualization parameters configuration. The system is, then, a more energy efficient and accessible version of the one produced by the Complutense University of Madrid and the École Polytechnique Fédérale de Lausanne, which is the primary objective of the project.\\

		This achievement is made thanks to the successful outcome of both the hardware research and the software development processes in which the project has been divided. Each of them requiring the employment of specific methodologies and techniques, but being, as they were, highly dependant one on the other only complicated the predicition of the outcome of them both. Thanks to the flexible scheduling conducted for each one, which considered the potential eventualities to arise in the other and focused in allowing rescheduling when necessary, this uncertainty has been correctly managed, leading to the current, successfully finalized state of the project.\\
		
		% HW: The goal of an external 802.15.4 receiver device is fulfilled. Employing it the system is able to render ECG data emitted by a delineator node. Bluetooth is also available, as well as log saving and further reading.\\
		% No olvidar hablar de modificaciones al shimmer!
		Regarding the hardware research part, the main goal pursued was the production of the 802.15.4 USB receiver device for Android systems. This device has successfully evolved from the early stages of development where a prototyping board was employed to a custom developed printed circuit board, which, if has not been produced, is completely designed and validated\todo{Not yet, but soon}.\\

		% Concrete results: X x Y x Z sized device consuming W1 watts of power (actual data, please) which compared to Bluetooth consumption (W2 watts, ...) is pretty low cost, all of this powered by the Android device, and the receiver being plug and play no installation procedure is required.\\
		The prototype board's \todo{Substitute these for the PCB ones if available}dimensions are 7.25 x 6.35 x 3.5, and it requires 3.3V for correct operation, being usable with at least 3.0V. It's USB capability allows for it to be connected to any HID compliant system, and has been successfully employed as an 802.15.4 receiver both in Android devices and Windows based personal computers.\\

		In respect of the Android application, the software development project has produced an Android application providing all required functionality, namely visualization of ECG data from Bluetooth and USB nodes, log saving and reading, and view controlling capabilities, designed and implemented in a robust, expandable manner. Moreover, the application provides a domain-specific framework for the inclusion of new data sources (like Wi-Fi or Near Field Connection) or different source data specifications.\\

		The application presents simple, easy to use user interfaces and very specific functionality, which added to the following of Android proposed application design best-practices smooth the learning curve and allow out of the box usage of the software part of the system. Due to the expansion capabilities of the system, new user interfaces or dialogs can also be added with ease.\\
	
	\section{Potential additions and expansions}
	\label{sec:end-further}

		Potential additions are to target the software frontend application. The developed Android application is a (domain specific) general purpose monitoring frontend that should provide a solid framework for further developments.\\

		Professional multipatient monitoring could come in two flavours:\\

		Visual-less multipatient monitoring in which a computer receives data from a variable number of wireless delineation nodes, e.g. employing the 802.15.4 receiver, storing it and acting as a server, or directly sending the log files to the actual server. The Android device would then download the log from the server and the own frontend application developed in the scope of this project could be used as the visualization device.\\

		The other option is simultaneous multipatient monitorization in an Android device. The monitorization application on the device would allow switching between patient ECG wave visualization while logging all received data, which could then be uploaded to a server.\\

		(Both of these expansions could find an employment in acutal medical environment.)


		More improvements can include the implementation of more detailed log navigation functionality, including information about the actual recording time and searching of specific time moments.
		Inclusion of event data into the log (like body weakness sensation or feeling of dizziness) could also be useful.
		(This two are for personal monitorization and inhome healthcare)\\

		Message sending when certain events occur (low or high hbr or arrythmia detection), text message, email, even a phone call could help constant monitorization requiring people. GPS information could be included in the message for quick localization of the affected person by the healthcare personal.
	
	\section{Findings}
	\label{sec:end-findings}

		Ideas: importance of monitorization systems for cardiovascular diseases affected people. Great potential of development in this area. Low cost, low sized, user focused designed, an thus, comfortable application system development is a ineherntly good goal, as are of great help for CVD affected people.

		Interesting: Marian is interested in the project for personal monitorization, David Atienza (EPFL's ESL \& UCM) is highly interested in the lower energy costs this project has achieved. Also, Solar Flight.
