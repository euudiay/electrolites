\chapter{Hardware and communications}
\label{ch:hardware}
	\section{Overview}	
	%que necesidad cubre la parte hardware del proyecto
	The hardware in our project cover a main need, an external device to be able to communicate our android device with a *shimmer through 802.15.4. Than device have to be a little and low powered device that can be conected as device through USB in an android device. Little, because a device that disturb a regular work is not valid at all. Low powered, because if the cost of power the divece is higher than use the stack bluetooth we lose an important adventage of use 802.15.4. Able to be coneced throught USB in our android device because this is the only way to interact with android for a external device. And finally able to be connected as device to elimminate the needed of an extra batery that would have incresed the cost and size of our device. \\

	In order to achieve this ambitious goal, we divide this develop in milestones that will help us to focus our works in more concrete tasks and correctly finish the project. \\

	Before of introduce more infromation about our project we have a section of technologies that will be very usefull to understand all this chapter and that will be referenced many times in other sections

	\section{Researched Tecnologies}
	%mini introducción diciendo que para la correcta comprensión de está sección surge la necesidad de explicar en mayor o menor profundidad las siguientes tecnologías usadas en algún punto del desarollo del proyecto.
		\subsection{Arduino}
		\subsection{MSP430}
		%Posibilidad de tras la epxlicación inicial meter dos secciones más que expliquen las diferencias entre los dos chips y las dos placas usadas(hablarlo todos)
		\subsection{802.15.4}
		\subsection{FreeRTOS}
		%Pedir información a recas y referencias a recas
		\subsection{USB device \& USB host}

	\section{Description}

	%Esta parte es el cuerpo de la parte de investigación del proyecto, no se sabía si se podía, no se sabía cómo hacerlo, …
	The hardware is the center section of the research in the whole project, not because there wasn't more research, but because nobody has researched this areas. At the begining we just know what we want to do but we have absolutely nor idea or clues about how we can do a very important number of our *milestones. In any cases we don't even know if our goals could be achieved, in particular a very important one, we need to connect a MSP430 to a Android where Android acts as host, and nobody achieve this before and therefore there are no information about that in forums or TI official support.\\

	%Plantea un reto porque toca todos los niveles, desde diseño de PCB a nivel componente hasta desarrollo a nivel de SO (kernel? <= investigar si kernel o SO)
	The project suppose a chalenge because it involves every hardware level, form the lower levels as the PCB design of a device to higher levels as develop parts of a SO. *Aqui molaria poner algo más que dos tristes lineas. \\
	
	Scarap zone.\\
	% USB device vs host. 
	As we explain in the subsection 3.2.5 *Como se ponen enlaces?* the host rol is assumed by who manage the connection, and device rol by the one who just use this connection to send and recive data and recive energy, for us, this minds a simpler programming in our MSP430 device, which code is harder to develop than android code.\\

	%MSP430 interacción con android
	The interaction between MSP430 and android was the most dangerous risk of this develop because there aren't information of any kind because it's not researched. USB is quite not as simple as everybody believes, there are many different protocols that works with USB, and each protocol can be implemented in very different ways, this implies a huge investigation process to find if any of them can be used by us. The final way to communicate both devices will be extendedly explained en section 3.4.2.\\

	%Bluetooth vs ZigBee(NO, 802.15.4). A lo mejor es suficiente con lo de la introducción(estudiar esa sección)
	We decided to use 802.15.4 instead of Bluetooth for many reasons, like lower consums or aviability to create networks to cover big surfaces. But the more important one is that *shimmer have a very small battery that, with bluetooth just lasts 5 or 6 hours, but with 802.15.4 it colud lasts more than a week. Talking about the delineator device that have to be carried by the patient, the difference between charge the device 3 or 4 times every day and charge it less than once every week is more than significant.\\

	%Con todo esto en mente y bla bla se plantean los siguientes hitos
	*Parrafo para introducir los hitos en el que no tengo ni idea que decir, viva!*	
	
	\section{Milestones}	

		\subsection{Arduino for Android USB Device Comunication}
		%Prototipo sencillo que sirvió para familiarizarnos con la programación de USB en android, rápido porque la parte del arduino estaba hecha, y programar USB device en Android(referencia a algúna parte del software donde se cuente como de facil se implementa eso) no supone demasiada complicación	

		\subsection{MSP430 for Android USB Host Comunication}
		%Líneas de investigación(plantear de manera lo más cronológica posible remarcando que aui estabamos totalmente a ciegas, no sabíamos como ibamos a alcanzar nuestro objetivo, solo teníamos una API USB proporcionada por TI para comunicación USB con windows: 
			%Driver propio para linux 
			%Driver propio para android (ambos sobre driver TI MSP430 para MAC)
		%[Mencionar Riesgos asumidos al abandonar línea driver propio y seguir investigando]
		%Encontramos mención a driver genérico HID en android + TI’s HID api for MSP430s
		%Probamos y funcionó (más trabajo software, enlazar a capítulo) , comentar la importancia del hardware en esta parte de la programación android debido a que ese código es muy dependiente de como implemente el protocolo HID(que proporciona mucha libertad)el MSP lo que supuso mucha investigación y continuo acceso a manuales tanto de la API como del propio MSP(ese comentario iria aqui o en la parte de software?Hablarlo todos)


		\subsection{MSP430 and FreeRTOS}	
		%Adaptación de la API para hacerla funcionar en un SO basado en tareas como es el FreeRTOS
		%Importantes riesgos de conflictos a nivel de compartición de recursos hardware, que con cuidado(más con suerte que con cuidado) pudieron ser esquivados.
		
	
		\subsection{802.15.4 in FreeRTOS}
		%Radio probada en una placa que no proveía USB pero sí salida por puerto serie para simplificar trabajo y facilitar la depuración, se llevó luego a otra con USB pero sin soporte ni software ni hardware para la radio, obligando a mapeo manual de pines, que ahora si con más mañana que suerte se hizo bien y no dio conflicto como veremos ahora aunque si hubo que modificar algo más de código dado que la nueva placa necesitaba iniciar los pines de la radio de otra forma.
		%FreeRTOS, parte radio en desarrollo por parte de terceros: trabajo incluyó buscar fallos, arreglarlos, terminar lo inacabado y aplicarlo al objetivo; en colaboración con el autor de FreeRTOS 

		\subsection{802.15.4 \& USB coexistence under FreeRTOS}
		%Riesgos de conflicto hardware entre Radio y USB
		%Riesgos de conflicto software entre Radio y USB sobre FreeRTOs
			%(Falta de tiempo para enviar y recibir)
		%Darle mucho peso a los riesgos y ver como tratar el tema de que no hubo prácticamente ningún problema con ellos(solo al de tiempo para enviar y recibir) sin que se note que este punto no tuvo mucho peso

		\subsection{MSP430 based device design}
		%Algo en plan, finalizada la investigación y gracias a que pusimos mucho empeño en acabar con tiempo suficiente fuimos capaces(como de tener la capacidad, no de conseguir(be able to)) de diseñar un dispositivo a partir de los componentes presentes en la placa de prototipado.
		%Identificación de componentes de placa 
			%(pines de expansión)
			%(cambio de componentes por otros)
			%(eliminacion JTAG y reutilizacion Radio)
		%Captura de esquemático
		%Diseño PCB
		%Rutado PCB

		\section{Final Product}
		%Descripción de que es lo que hemos conseguido(es posible que con la evolución alguien que se lea las iteraciones no consiga una clara visión de cual es el estado final del dispositivo, por lo que esta parte me parece muy importante)
		%Tener cuidado de que no haya algo muy parecido en la parte de la introducción, aunque si debiera un miniresumen del producto, lo que de la conclusión debería incluir tambien posibles usos no contemplados de lo que hemos conseguido.
