\chapter{Resumen en español}
\label{ch:resumen}

	En este proyecto se expone la investigación realizada con el objetivo de desarrollar un receptor siguiendo el estándar de redes de área personal inalámbricas 802.15.4 del IEEE para dispositivos Android a través de USB, aplicado a un sistema completo de monitorización electrocardiográfica (ECG), así como el proceso de desarollo del mismo. Este sistema se enmarca en el ámbito de la atención sanitaria personal: su principal aplicación es la montorización del estado del corazón por parte de un particular, eliminando la dependencia, respecto a esta tarea específica, con los sistemas de atención sanitaria tradicionales.\\

	Recientemente ha surgido un gran interés, tanto en el ámbito académico como en el industrial, en la producción de sistemas de monitorización ECG portátiles y de bajo consumo, llegando a ser una de las principales aplicaciones de las redes de sensores corporales inalámbricas.\\

	Para maximizar la portabilidad, en el desarrollo de estos sistemas se han empezado a emplear dispositivos móviles de gran capacidad de cómputo, particularmente smartphones, debido a la gran difusion que han tenido en los últimos años. En el 2011  se presentó un sistema, colaboración entre la Universidad Complutense de Madrid (UCM) y la École Polytechnique Fédérale de Lausanne (EPFL), para monitorización ECG de ámbito personal empleando un iPhone como visualizador y Bluetooth como tecnología de comunicación inalámbrica\todo{Citas en el resumen?}.\\

	De los resultados obtenidos por ese proyecto surge la presente iniciativa que trata de llevar al siguiente nivel las características inherentemente buenas de bajo consumo y bajo coste del mismo mediante la aplicación del protocolo 802.15.4, de mucho menor consumo energético que la tecnología Bluetooth, y la sustitución del dispositivo iOS por uno basado en Android, debido a su mayor accesibilidad y el menor coste, en general, de éstos.\\

	El sistema a desarrollar en este proyecto presenta al usuario una representación visual de su onda ECG en tiempo real, resaltando puntos relevantes para simplificar la compresión de los datos mostrados. También muestra información sobre el ritmo cardiaco, y toda esta información es almacenada de forma transparente al usuario para su posterior consulta.\\

	Esta funcionalidad es posible gracias a la operación conjunta de los tres dispositivos que forman el sistema de monitorización: el nodo de delineación ECG, el receptor 802.15.4 y el dipositivo Android que actúa de interfaz con el sistema. El nodo de delineación ECG va conectado a la red de sensores corporal del usuario y se encarga de la captura y posterior análisis de la onda ECG, así como de la codificación y envío de la misma de forma inalámbrica. El receptor 802.15.4 conectado al sistema Android a través de USB controla la recepción de datos a traves de dicho protocolo y el envío de la información recibida al dispositivo Android. Éste actúa como decodificador y visualizador en tiempo real, y como interfaz con el sistema, gestiona las conexiones inalámbricas y almacena y muestra los datos recibidos.\\

	Los objetivos del proyecto son, entonces, el desarrollo de la aplicación para dispositivos Android, la producción del receptor 802.15.4 y la comunicación de ambos con un nodo delineador ECG ya existente.\\

	La aplicación para dispositivos Android, como ya se ha mencionado, es la interfaz con la que el usuario interactúa con el sistema. Su diseño sigue las prácticas comunes de aplicaciones para este sistema operativo ya que el objetivo es que la curva de aprendizaje sea, si no nula, muy suave. Los motivos para emplear Android como sistema operativo base son tres: la importancia de éste entre los sistemas operativos móviles, el mayor rango de precios de los terminales que lo soportan, que permite una mayor difusión del sistema debido a la existencia de dispositivos de precio más reducido, y la naturaleza libre y de código abierto del entorno de desarrollo, característica que facilita la posterior expansión del sistema. Todos estos motivos pueden resumirse en que el empleo de Android como sistema operativo permite el acceso de un mayor número de usuarios al mismo.\\

	En cuanto al nodo delineador de la onda ECG, el objetivo es emplear uno ya desarrollado para el proyecto. Esto es así porque tanto el nodo delineador como la red de sensores corporales que capturan la onda ECG son sistemas especialmente complejos cuyos desarrollos ocuparían un proyecto de la envergadura del actual cada uno.\\

	El nodo delineador que se emplea en el proyecto es obtenido en el proyecto de la UCM y la EPFL antes mencionado. Este nodo se desarrolló inicialmente como 