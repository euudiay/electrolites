\chapter{Results}
\label{cha:results}

	\section{Final state}
	
		% Describes the fulfillment of this project.
		Podemos usar esto para describir exactamente lo que hemos conseguido con datos de consumos o lo que haga 				falta.

		Lo tipico, hemos conseguido un sistema capaz de recibir por radio en un dispositvo externo datos de ECG que despues se pasan a un tablet que los muestra convenientemente y que además de eso tambien puede recibir esos mismos datos por bluetooth y mostrarlos para la posterior visualización de manera estática.

		Y aqui resultados concretos, tenemos un aparato de X:Y:Z dimensiones, que supone X1 gasto de batería al tablet mientras que el bluetooth le supone X2, y que al ser USB device no requiere de batería, que es plug-and-play, por no hace falta instalarlo tu lo conectas y eso va.

		Tenemos una aplicación robusta y facilmente expandible que nos permite conectarnos por USB al dispositivo de antes, conectarnos por bluetooth directamente al shimmer, o guardar y cargar los logs de todas las monitorizaciones anteriores de una forma limpia y muy intuitiva.

		Estas dos partes unidas a un componente externo como es el shimmer forman un sistema completo de monitorizacion en tiempo real comodo,  de muy bajo consumo y ultra portable.
	
	\section{Potential additions and expansions}

		Las principales expansiones serían a nivel de usabilidad y de la aplicación del software, que es la que más 					posibilidades tiene.

		Como mejora orientada a si quisiesemos expandirnos a hospitales se podría hacer que con el uso de otro MSP conectado a un servidor y con una aplicación muy muy sencilla se fuera loggeando todo en un ordenador a parte de lo que llega al tablet concreto que esté cerca. 

		Se podría hacer que la aplicación android se pudiese conectar con el servidor para cargar de ahí los logs, teniendolos así centralizados y accesibles para todo el hospital y no solo el tablet concreto que los ha recogido.

		Se podrían incluir facilidades de navegación por los logs como poner la hora a la que corresponde el segundo de onda que estás viendo actualmente, o incluso moverse por los logs a partir de la hora.

		Dado que es un tablet o movil se podría incluir soporte para que en caso de subidas o bajadas de tension(por ejemplo) mandase un mensaje a un numero o un correo a una cierta direccón.

		Se podría incluir localización por GPS para que en caso de que hubiera algún problema en el aviso que podría mandarse la localización por si el paciente no pudiese facilitarla en caso de que le llamaran para ver si está bien.
	
	\section{Findings}

	Hablar sobre la imporancia de los sistemas de monitorización para mucha gente, de que es algo con un uso potencial muy grande, y que nosotros venimos a cubrir esa necsidad con un sistema que ha tenido en cuenta la facilidad de uso por parte del usuario desde el princpio orientandolo por ello a sistemas de bajo consumo para que las personas que tienen algun problema cardiaco no tengan una nueva carga, si no una ayuda.