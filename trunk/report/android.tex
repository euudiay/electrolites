\chapter{Software Development} % application?

	% Brief description and scope for this chapter	

	\section{Overview}

	\begin{comment}
		Microestado del arte:
		Desarrollo para dispositivos android, paradigma particular, no estamos formados en él (y esto ha dado problemas), 
arquitecturas muy particulares, en el momento de comenzar el desarrollo documentación buena pero muy técnica, más para consulta que para formación. Versiones de android para usb host, … => impone requisitos al dispositivo tablet
    (Posibilidad de hilos destruidos en cada momento, atender al giro de pantalla, destrucción de la actividad, …)
    Limitaciones de android como plataforma (java vm, opengl, …)
    Aplicación iphone: funcionalidad limitada, captura de requisitos comenzó por ella, crear un producto a partir del prototipo.
    Se añadió feedback de los médicos con que trabaja Fran en Murcia (Preguntar a Recas) (en particular los logs!)

	\end{comment}

	% Introduction: app + feedback medical staff (hey, it's an important project!)
	The development of a software application targeted at Android Operating System for mobile devices is the counter-part to the hardware research part of the project.
	This application was to substitute the already developed one for iOS devices, adding funcionality extracted from feedback obtained from actual medical staff [!]Fran and EPFL[!]. The software must provide functionality to visualize ECG data from Bluetooth or 802.15.4 sources (the latter obtained via [!]our receiver node[!]) in realtime, as well as to save that data into file logs for afterwards reading.\\

	% Android general
	Android as a development platform provides a wide set of high abstraction level tools to emphasize robust and reusable design for low resource based, quick development cycles. Such benefits require the adequation of the software design and architecture to the constrains imposed by the Android development framework.\\

	% None android formation + android peculiarities
	Given that none of the project team members had received any instruction on this framework, engaging the development of an Android application implied an important risk. Moreover, after the research and training steps concluded, follow up of that risk was not halted, as the quick, robust software development is only assured when building an standard Android application; dynamic, soft real-time functionality implementation is not discouraged, but also not guaranteed to work.
	Mobile devices development restrictions and common practices were also unkown to the team.\\

	% Android limitations
	Even when the aforementioned eased development features are applicable, mobile devices are harsh software environments due to, amongst others, memory and battery constrains, where processes have to handle being suspended by an incoming call or similar external events. This factors are specially critical for an application as the one developed in this project, which needs to continually parse and log data.\\

	% App linked to hw development and a useful tool
	The application was also intended to act as a quick testing front-end for the prototypes produced by the parallel-conducted hardware research. By providing fully-functional application modules since early stages of development, hardware prototypes could be best-case and worst-case checked by directly connecting them to the Android device for data visualization. Visual verification proved to be a very effective method when working with large quantities of data which were more easily checked against their visual representation than value-by-value reading.\\

	% Development process
	These factors lead to the adoption of an agile software development process focusing on functionality building while prototyping more high risk involving features. To avoid typical drawbacks of such methodologies, great emphasis was put on the application of characteristics found in \textit{Iterative and Incremental processes}, namely, use case driven and risk focused development. That way, project scheduling was done addressing higher risks first while assuring expected functionality to be implemented on time thanks to the use case model.\\

	% Conclusion and chapter presentation
	In the following sections a complete view of the software development project will be presented, beginning with the requirements captured for the project. The use case scenarios identified from those requisites will be detailed next, followed by an explanation of the system architecture ?via 4+1 view model?. Then implementation details will be exposed and the chapter will finish with a short conclusion.

	\begin{comment}
	Full implementation, architectural and yadda yadda are presented in Annex X
	\end{comment}

	\section{Requirements}

		\begin{itemize}
		\item R01 - Receive raw data via Bluetooth
		\item R02 - Receive raw data via 802.15.4
		\item R03 - Receive raw data from a log file
		\item R04 - Parse raw data into processed data
		\item R05 - Display processed data
		\item R06 - Log raw data
		\item R07 - Log processed data	% Delete if not developed
		\item R08 - Scale View Vertically
		\item R09 - Scroll View Vertically
		\item R10 - Scroll View Horizontally
		\end{itemize}

	% ?Non-functional requirements?
	% 30fps
	% Android 3.1

	\section{Risk Analysis}

		Being the project mainly a hardware research project, and considering the software development part of it useless without successful results on the hardware part, a detailed process of risk analysis was mandatory to be conducted since the earlier stages of planning and development so as to avoid wasting manpower on futile work.\\

		The risk list at the end of the project is as follows:
		\begin{itemize}
		\item \textbf{PR1.} Application funcionality inferior to that featured by existing iOS application
		\item \textbf{HR1.} 802.15.4 receiver device unfeasible
		\item \textbf{HR2.} 802.15.4 receiver device delayed
		\item \textbf{AR1.} Android providing subpar performance when handling required data
		\item \textbf{AR2.} Android ?Xoom? rendering capabilities unable to handle required data
		\item \textbf{AR3.} Lack of instruction on Android development
		\end{itemize}

		% Risk anaylisis process explanation (decisions, ...)
		This risk anaylisis focused on two main risk sources: the parallel-conducted hardware research, and Android as a development platform. Project definition and team related risks were also considered.\\

		The hardware research part of the project delivered the highest probability and impact rated risks. It was so because those risks were external to the software development project scope and thus could not be handled by any of the tools provided by any development methodology. At the same time, should such risks come to be, the impact on the software product would be, in most of cases, as cathastrophic as turning the whole development useless thus causing it's cancellation.\\
		
		Regarding Android development only a subset of the final set of risks was assessed at first. Every risk in this subset dealt with the team lack of knowledge about the Android platform and was scheduled to be addressed foremost. A last risk was added to this group after the first research on mobile devices limitations regarding potential unfitness of such devices for near real-time display and handling of not-so-small data packages, and that risk handling plan proved to be key to the successful outcome of the project as the remaining subset of Android-related risks were linked to Android applications display performance.\\ % Further explanation on this last set?

		The usual project definition and personal risks such as incorrect deadline scheduling or unability to reach critical milestones on time were pondered, increasing their impact rates as the application would be needed by the hardware device to secure a successful outcome for the project.
		
		% Risk Table including evolution
		A detailed view of each assessed risk is provided next, including risk evolution throughout the project lifetime.

		\paragraph{HR1} Application funcionality inferior to that featured by existing iOS application\\
		\textbf{Probability:} Moderate\\
		\textbf{Impact:} Hight\\
		Two versions: 
			a. Hardware part and/or USB communication not achieved,
			b. Funcionality provided by iOS app to best not surpassed
			(or both)
		That makes the software project a failure.

		% Conclussion:
		\paragraph{draft} Thanks to this a schedule was developed that prioritized risk supressing and the decission was taken to plan only the first two of the five intended development iterations, leaving the other three as drafts to allow them to evolve at par with the uncancelled risks.
	
% Command to auto-generate Use Case headings (NON FUNCTIONAL YET)
\begin{comment}
\newcounter{uccounter}
\newcommand{\addUC}[1]{
	\stepcounter{uccounter}
	\bf{UC\arabic{uccounter} #1}
	}
\end{comment}
	
	\section{Use Cases}
		\subsection{UC1. View data from Bluetooth}

			\paragraph{Description} The user wish to receive and visualize data from a Bluetooth ECG node in real time. He will start the communication, visualize real-time received data and finish the connection once done.\\
			\\This use case captures requisites R01, R04, R05, R06, and R07.

			\paragraph{Preconditions} The application is in the main menu screen.
			\paragraph{Main flow}
				\begin{enumerate}
				\item The user indicates his will to start Bluetooth data visualization.
				\item The system prompts for the node to connect to.
				\item The user specifies the desired node.
				\item The system manages connection to the node. If unable to establish the connection, see AF1.
				\item The system shows processed data to the user. Received data is also logged.
				\item The user can now adjust view parameters (See UC4)
				\item The user chooses to finish data visualization.
				\item The system closes active connections and stops data visualization.
				\item The system returns to the main menu.
				\end{enumerate}

			\paragraph{Alternative Flow 1} The system cannot establish connection to the Bluetooth node selected by the user.
				\begin{enumerate}
				\item The system notifies the user about the problem.
				\item The system returns to the main menu.
				\end{enumerate}

		\subsection{UC2. View data from USB Receiver}

			\paragraph{Description} The user wish to receive and visualize data from 	an 802.15.4 ECG node in real time. He will start the communication, visualize real-time received data and finish the connection once done. The data from the node will be received via the USB 802.15.4 receiver device.\\
			\\This use case captures requisites R02, R04, R05, R06, and R07.

			\paragraph{Preconditions} The application is in the main menu screen.
			\paragraph{Main flow}
				% Aquí evitaría el uso de tanto "the", me suena mejor "User indicates..." y "System manages..."
				\begin{enumerate}
				\item The user indicates his will to start USB receiver data visualization.
				\item The system asks the user to connect the USB receiver.
				\item The user connects the USB receiver.
				\item The system manages connection to the USB receiver. If unable to establish the connection, see AF1.
				\item The system shows processed data to the user. Received data is also logged.
				\item The user can now adjust view parameters (See UC4)
				\item The user chooses to finish data visualization.
				\item The system closes active connections and stops data visualization.
				\item The system returns to the main menu.
				\end{enumerate}

			\paragraph{Alternative Flow 1} The system cannot establish connection to the USB receiver device.
				\begin{enumerate}
				\item The system notifies the user about the problem.
				\item The system returns to the main menu.
				\end{enumerate}

		\subsection{UC3. View data from log file}

			\paragraph{Description} The user wishes to read a log file created from a real time visualization session. He will specify the log file to load, visualize stored data and finish visualization once done.\\
			\\This use case captures requisites R03, R04 and R05.

			\paragraph{Preconditions} The application is in the main menu screen.
			\paragraph{Main flow}
				\begin{enumerate}
				\item The user indicates his will to start log data visualization.
				\item The system prompts for the log file to load.
				\item The user specifies the desired file.
				\item The system reads the selected log file.
				\item The system shows logged data to the user.
				\item The user can now adjust view parameters (See UC4)
				\item The user chooses to finish data visualization.
				\item The system stops data visualization.
				\item The system returns to the file selection menu.
				\item The user selects to return to main menu. Else follow from step 3.
				\item The system returns to the main menu.
				\end{enumerate}

		\subsection{UC4. Adjust view parameters}

			\paragraph{Description} When visualizing ECG data the user wishes to adjust view parameters such as plot vertical scale, plot vertical scroll and plot horizontal scroll.\\
			\\This use case captures requisites R08, R09, R10.

			\paragraph{Preconditions} The application is displaying ECG data.
			\paragraph{Main flow}
				\begin{enumerate}
				\item The user indicates his will to change the vertical scale.
				\item The system updates plot vertical scale.
				\item The user indicates his will to change plot vertical scroll.
				\item The system updates plot vertical scroll.
				\item If the displayed data is read from a log file, see AF1.
				\end{enumerate}
			
			\paragraph{Alternate Flow 1} The user is able to control horizontal scroll parameter.	
				\begin{enumerate}
				\item The user indicates his will to change the horizontal scroll.
				\item The system updates plot horizontal scroll.
				\end{enumerate}

	\section{Design and Architecture}

	\section{Implementation Details}

		\begin{comment}
			Potential iterations
			It1. Bluetooth + Architecture prototype
			It2. USB + Log
			It3. Architecture rewriting
			It4. Adding it all and finishing
		\end{comment}

		First two iterations planned so as to reach iOS software functionality, next iterations only drafted because dependant on hw research.

		\subsection{Iteration 1}

			The main objectives for this first iteration were
			\begin{itemize} 
				\item the instruction of the team on Android development, 
				\item the lay out of an initial version of the application architecture and
				\item the implementation of the Bluetooth receiver module.
			\end{itemize}

			Learning of android, product disposable.
			Architecture as a prototype to evolve to (or be substituted by) next versions.
			By the end of the iteration, UC1 realization was complete but not final.
			Not yet hardware development => full team at this => smaller time
			Bluetooth module fully developed except nice user interface and overall user friendlyness

			Results and adaptation of pre-planned it2
			
		\subsection{Iteration 2}

			The main objectives for this first iteration were
			\begin{itemize} 
				\item the instruction of the team on Android development, 
				\item the lay out of an initial version of the application architecture and
				\item the implementation of the Bluetooth receiver module.
			\end{itemize}
			As can be noted, this
	\section{Closure}
