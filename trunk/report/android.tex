\chapter{Software Development} % application?

	% Brief description and scope for this chapter	

	\section{Overview}

	\begin{comment}
		Microestado del arte:
		Desarrollo para dispositivos android, paradigma particular, no estamos formados en él (y esto ha dado problemas), 
arquitecturas muy particulares, en el momento de comenzar el desarrollo documentación buena pero muy técnica, más para consulta que para formación. Versiones de android para usb host, … => impone requisitos al dispositivo tablet
    (Posibilidad de hilos destruidos en cada momento, atender al giro de pantalla, destrucción de la actividad, …)
    Limitaciones de android como plataforma (java vm, opengl, …)
    Aplicación iphone: funcionalidad limitada, captura de requisitos comenzó por ella, crear un producto a partir del prototipo.
    Se añadió feedback de los médicos con que trabaja Fran en Murcia (Preguntar a Recas) (en particular los logs!)

	\end{comment}

	% Introduction: app + feedback medical staff (hey, it's an important project!)
	The development of a software application targeted at Android Operating System for mobile devices is the counter-part to the hardware research part of the project.
	This application was to substitute the already developed one for iOS devices, adding funcionality extracted from feedback obtained from actual medical staff [!]Fran and EPFL[!]. The software must provide functionality to visualize ECG data from Bluetooth or 802.15.4 sources (the latter obtained via [!]our receiver node[!]) in realtime, as well as to save that data into file logs for afterwards reading.\\

	% Android general
	Android as a development platform provides a wide set of high abstraction level tools to emphasize robust and reusable design for low resource based, quick development cycles. Such benefits require the adequation of the software design and architecture to the constrains imposed by the Android development framework.\\

	% None android formation + android peculiarities
	Given that none of the project team members had received any instruction on this framework, engaging the development of an Android application implied an important risk. Moreover, after the research and training steps concluded, follow up of that risk was not halted, as the quick, robust software development is only assured when building an standard Android application; dynamic, soft real-time functionality implementation is not discouraged, but also not guaranteed to work.
	Mobile devices development restrictions and common practices were also unkown to the team.\\

	% Android limitations
	Even when the aforementioned eased development features are applicable, mobile devices are harsh software environments due to, amongst others, memory and battery constrains, where processes have to handle being suspended by an incoming call or similar external events. This factors are specially critical for an application as the one developed in this project, which needs to continually parse and log data.\\

	% App linked to hw development and a useful tool
	The application was also intended to act as a quick testing front-end for the prototypes produced by the parallel-conducted hardware research. By providing fully-functional application modules since early stages of development, hardware prototypes could be best-case and worst-case checked by directly connecting them to the Android device for data visualization. Visual verification proved to be a very effective method when working with large quantities of data which were more easily checked against their visual representation than value-by-value reading.\\

	% Development process
	These factors lead to the adoption of an agile software development process focusing on functionality building while prototyping more high risk involving features. To avoid typical drawbacks of such methodologies, great emphasis was put on the application of characteristics found in \textit{Iterative and Incremental processes}, namely, use case driven and risk focused development. That way, project scheduling was done addressing higher risks first while assuring expected functionality to be implemented on time thanks to the use case model.\\

	% Conclusion and chapter presentation
	In the following sections a complete view of the software development project will be presented, beginning with the requirements captured for the project. The use case scenarios identified from those requisites will be detailed next, followed by an explanation of the system architecture ?via 4+1 view model?. Then implementation details will be exposed and the chapter will finish with a short conclusion.

	\begin{comment}
	Full implementation, architectural and yadda yadda are presented in Annex X
	\end{comment}

	\section{Requirements}

		% Paragraph here explaining requirement capture process
		\subsection{Functional Requirements}

		\begin{itemize}
		\item R01 - Receive raw data via Bluetooth
		\item R02 - Receive raw data via 802.15.4
		\item R03 - Receive raw data from a log file
		\item R04 - Parse raw data into processed data
		\item R05 - Display processed data
		\item R06 - Log raw data
		\item R07 - Log processed data	% Delete if not developed
		\item R08 - Scale View Vertically
		\item R09 - Scroll View Vertically
		\item R10 - Scroll View Horizontally
		\end{itemize}

	% ?Non-functional requirements?
	% 30fps
	% USB-host Android 3.1

		\subsection{Non-functional Requirements}

		The following non-functional requirements are identified:
		\begin{itemize}
			\item The application must display ECG data at 30fps.
			\item The application must run on a Motorla Xoom device.
		\end{itemize}

	\section{Risk Analysis}

		Being the project mainly a hardware research project, and considering the software development part of it useless without successful results on the hardware part, a detailed process of risk analysis was mandatory to be conducted since the earlier stages of planning and development so as to avoid wasting manpower on futile work.\\

		The risk list at the end of the project is as follows:
		\begin{itemize}
		\item \textbf{PR1.} Application funcionality inferior to that featured by existing iOS application

		\item \textbf{HR1.} 802.15.4 receiver device delayed
		\item \textbf{HR2.} 802.15.4 receiver device unfeasible

		\item \textbf{AR1.} Lack of instruction on Android development delays workflow
		\item \textbf{AR2.} Android providing subpar performance when handling required data
		\item \textbf{AR3.} Android rendering capabilities unable to handle required data

		\item \textbf{MR1.} Mobile device unsuitable for target functionality
		\end{itemize}

		% Risk anaylisis process explanation (decisions, ...)
		This risk anaylisis focused on two main risk sources: the parallel-conducted hardware research, and Android as a development platform. Project definition and team related risks were also considered.\\

		The hardware research part of the project delivered the highest probability and impact rated risks. It was so because those risks were external to the software development project scope and thus could not be handled by any of the tools provided by any development methodology. At the same time, should such risks come to be, the impact on the software product would be, in most of cases, as cathastrophic as turning the whole development useless thus causing it's cancellation.\\
		
		Regarding Android development only a subset of the final set of risks was assessed at first. Every risk in this subset dealt with the team lack of knowledge about the Android platform and was scheduled to be addressed foremost. A last risk was added to this group after the first research on mobile devices limitations regarding potential unfitness of such devices for near real-time display and handling of not-so-small data packages, and that risk handling plan proved to be key to the successful outcome of the project as the remaining subset of Android-related risks were linked to Android applications display performance.\\ % Further explanation on this last set?

		The usual project definition and personal risks such as incorrect deadline scheduling or unability to reach critical milestones on time were pondered, increasing their impact rates as the application would be needed by the hardware device to secure a successful outcome for the project.\\
		
		% Risk Table including evolution
		A detailed view of each assessed risk is provided next, including risk evolution throughout the project lifetime.\\

		% PR1
		\paragraph{PR1.}Application functionality inferior to that featured by existing iOS application\\
		\textbf{Probability:} Moderate\\
		\textbf{Impact:} Very High\\
		\textbf{Description:} Failure to provide an expanded set of features in the Android application when compared with the iOS application renders the software part of the project invalid on its own. It could, then, only be valid as demo software for USB receiver device showcasing. If the device is nor finished, then the whole software development project will have been futile.
		The key marker for this risk is unability to generate valid software modules throughout the development that provide required functionality. Failure to reach milestones and use case realizations on time is other important marker.
		Preventive measures were taken to avoid the occurence of this risk since the beginning of the development by a functionality building focused project scheduling for the first development phases.\\
		\textbf{This risk was marked as surpassed at the reviewing metting of Iteration 2 as all key functionality had been implemented, as planned.}

		% HR1
		\paragraph{HR1.}802.15.4 receiver device delayed\\
		\textbf{Probability:} High\\
		\textbf{Impact:} High\\
		\textbf{Description:} Being the production of the 802.15.4 receiver device dependant on the hardware research part of the project a delay on the estimated milestones for that part of the project is likely to occur. Should that happen, hardware research and development will need to be prioritized over this software project. That could lead to big delays in software production.
		To prevent the rising of further problems derived from those potential delays, the software development process must always work with non-solid, ready-to-change deadlines and milestones. Application functionality is to be ranked in order of importance of implementation to be prepared, in case of an unexpectedly big delay, to leave less important functionality out of the scope of the project.
		Markers to be followed up are: unsuccessful output from hardware research (a new branch of the potential technologies tree has to be explored), failure to reach hardware development or research milestones and delays in the acquisition of tools or devices needed for the hardware project.
		Preventive measures considered are: detailed follow up of the hardware research development, reducing the software development team if manpower is needed in the hardware area, and planning asuming delays on component acquisition.\\
		\textbf{HR1 was monitored throughout the whole software development project, and marked as surpassed at the reviewing meeting of Iteration 5.}

		% HR2
		\paragraph{HR2.}802.15.4 receiver device unfeasibe\\
		\textbf{Probability:} Medium\\
		\textbf{Impact:} Critical\\
		\textbf{Description:} Until hardware research results are successfully delivered there is no guarantee of the viability of the 802.15.4 receiver device. This software development project loses most of its value if such device is not developed, as the iOS application already exists. Developing an Android application with an equal feature set is also a valid objective, so this risk does not render the development invalid: the full team will then work on software development, and requirements will be restated to include more final-user oriented functionality and/or features from the \emph{future} set.
		This risk can be monitored with the following markers: unsuccessful output from hardware research and failure to reach hardware development or research milestones. Being both external to this software project, no preventive measures can be applied apart from scheduling allowing smaller team sizes for the software area.\\
		\textbf{The probability of the risk was reduced to low after the reviewing meeting for Iteration 3, when the critical hardware research had concluded with positive results. HR2 was marked as surpassed when the production of a device prototype was finished and tested [?TIME?].}

		% AR1
		\paragraph{AR1.}Lack of instruction on Android development delays workflow\\
		\textbf{Probability:} High\\
		\textbf{Impact:} Moderate\\
		\textbf{Description:} None of the team members has received any instruction on Android development and throughout research is not viable because of time restrictions. It is reasonable to foresee potential delays in the development because of the parallel instruction-development flow, as well as the need to rewrite parts of the system rendered obsolete when further knowledge is acquired.
		Application malfunctioning, unexpected behaviours and low performance are markers to be tracked.
		As a preventive measure a short instruction time will be scheduled at the beginning of the project, but every team member is responsible to continue his instruction throughout the whole project. Application builds are to be checked for big differences against canon Android applications behaviour.\\
		\textbf{The risk was marked as surpassed after Iteration 3, as critical functionality had already been implemented and tested, though Android instruction was not halted.}

		% AR2
		\paragraph{AR2.}Android providing subpar performance when handling required data\\
		\textbf{Probability:} Moderate\\
		\textbf{Impact:} High\\
		\textbf{Description:} The benefits of the high-level, single application model provided by Android are such in behalf of the sacrifice of performance. In this project soft-realtime requirements are present, and the system needs to process around 250 ECG wave samples [quotation needed] (among other data) per second. Android code reutilization and class based programming suggested practices, the absence of an explicit memory management API and the employment of the Garbage Collector only complicate the achievement of such requirements.  
		Special care will need to be put on the development and performance checks are to be conducted regularly on generated builds to ensure the avoidance of this risk.
		If evidence is found of Android inability to provide the required perfomance (and there is no way of attributting the failure to the team's lack of ability), low-level Android development will be considered. As the probability of this last scenario to occur is quite low, no research will be conducted in low-level Android development until mandatory.\\
		\textbf{The risk was verified to be occuring during Iteration 2 testing phase. Lack of care on memory management was found to be the problem, and was solved in Iteration 3. The risk was marked as surpassed after the review meeting for Iteration 4.}

		% AR3
		\paragraph{AR3.}Android rendering capabilities unable to handle required data\\
		\textbf{Probability:} Low\\
		\textbf{Impact:} High\\
		\textbf{Description:} The Android Operating System runs on quite a wide range of devices, each with it's own technical specifications. Providing the single application development model that Android features requires many software layers, many of them of high abstaction level. The risk exists, thus, that the rendering required by the project couldn't be achieved within the involved time restrictions. The target device for the project is fixed (see Non-functional Requirements Subsection [link!]) as a Motorola Xoom. This device employs a dedicated Tegra2 GPU [quotation needed] which should suffice, so risk probability is choosen as \emph{low}. Performance tests in the display module are to be conducted, though, so as to make sure that a correct usage of the available resources is being done. 
		If low rendering performance is detected, Android native level rendering API, Renderscript, is to be looked into as a remedy, once the code is assured to be optimized.
		\\
		\textbf{Risk probability was increased to \emph{Moderate} during Iteration 3 as low performance was detected but was considered not critical enough to apply the Renderscript solution. The risk was marked as surpassed in the reviewing meeting of Iteration 4.}

		% MR1
		\paragraph{MR1.}Mobile device unsuitable for target functionality\\
		\textbf{Probability:} ???\\
		\textbf{Impact:} ???\\
		\textbf{Description:} Risk explanation goes here.\\
		\textbf{Risk evolution goes here.}
		
		\begin{comment}
		Remaining:
		{MR1.} Mobile device unsuitable for target quantity of functionality
		\end{comment}

		% Conclussion:
		\paragraph{draft} Thanks to this a schedule was developed that prioritized risk supressing and the decission was taken to plan only the first two of the five intended development iterations, leaving the other three as drafts to allow them to evolve at par with the uncancelled risks.
	
% Command to auto-generate Use Case headings (NON FUNCTIONAL YET)
\begin{comment}
\newcounter{uccounter}
\newcommand{\addUC}[1]{
	\stepcounter{uccounter}
	\bf{UC\arabic{uccounter} #1}
	}
\end{comment}
	
	\section{Use Cases}
		\subsection{UC1. View data from Bluetooth}

			\paragraph{Description} The user wish to receive and visualize data from a Bluetooth ECG node in real time. He will start the communication, visualize real-time received data and finish the connection once done.\\
			\\This use case captures requisites R01, R04, R05, R06, and R07.

			\paragraph{Preconditions} The application is in the main menu screen.
			\paragraph{Main flow}
				\begin{enumerate}
				\item The user indicates his will to start Bluetooth data visualization.
				\item The system prompts for the node to connect to.
				\item The user specifies the desired node.
				\item The system manages connection to the node. If unable to establish the connection, see AF1.
				\item The system shows processed data to the user. Received data is also logged.
				\item The user can now adjust view parameters (See UC4)
				\item The user chooses to finish data visualization.
				\item The system closes active connections and stops data visualization.
				\item The system returns to the main menu.
				\end{enumerate}

			\paragraph{Alternative Flow 1} The system cannot establish connection to the Bluetooth node selected by the user.
				\begin{enumerate}
				\item The system notifies the user about the problem.
				\item The system returns to the main menu.
				\end{enumerate}

		\subsection{UC2. View data from USB Receiver}

			\paragraph{Description} The user wish to receive and visualize data from 	an 802.15.4 ECG node in real time. He will start the communication, visualize real-time received data and finish the connection once done. The data from the node will be received via the USB 802.15.4 receiver device.\\
			\\This use case captures requisites R02, R04, R05, R06, and R07.

			\paragraph{Preconditions} The application is in the main menu screen.
			\paragraph{Main flow}
				% Aquí evitaría el uso de tanto "the", me suena mejor "User indicates..." y "System manages..."
				\begin{enumerate}
				\item The user indicates his will to start USB receiver data visualization.
				\item The system asks the user to connect the USB receiver.
				\item The user connects the USB receiver.
				\item The system manages connection to the USB receiver. If unable to establish the connection, see AF1.
				\item The system shows processed data to the user. Received data is also logged.
				\item The user can now adjust view parameters (See UC4)
				\item The user chooses to finish data visualization.
				\item The system closes active connections and stops data visualization.
				\item The system returns to the main menu.
				\end{enumerate}

			\paragraph{Alternative Flow 1} The system cannot establish connection to the USB receiver device.
				\begin{enumerate}
				\item The system notifies the user about the problem.
				\item The system returns to the main menu.
				\end{enumerate}

		\subsection{UC3. View data from log file}

			\paragraph{Description} The user wishes to read a log file created from a real time visualization session. He will specify the log file to load, visualize stored data and finish visualization once done.\\
			\\This use case captures requisites R03, R04 and R05.

			\paragraph{Preconditions} The application is in the main menu screen.
			\paragraph{Main flow}
				\begin{enumerate}
				\item The user indicates his will to start log data visualization.
				\item The system prompts for the log file to load.
				\item The user specifies the desired file.
				\item The system reads the selected log file.
				\item The system shows logged data to the user.
				\item The user can now adjust view parameters (See UC4)
				\item The user chooses to finish data visualization.
				\item The system stops data visualization.
				\item The system returns to the file selection menu.
				\item The user selects to return to main menu. Else follow from step 3.
				\item The system returns to the main menu.
				\end{enumerate}

		\subsection{UC4. Adjust view parameters}

			\paragraph{Description} When visualizing ECG data the user wishes to adjust view parameters such as plot vertical scale, plot vertical scroll and plot horizontal scroll.\\
			\\This use case captures requisites R08, R09, R10.

			\paragraph{Preconditions} The application is displaying ECG data.
			\paragraph{Main flow}
				\begin{enumerate}
				\item The user indicates his will to change the vertical scale.
				\item The system updates plot vertical scale.
				\item The user indicates his will to change plot vertical scroll.
				\item The system updates plot vertical scroll.
				\item If the displayed data is read from a log file, see AF1.
				\end{enumerate}
			
			\paragraph{Alternate Flow 1} The user is able to control horizontal scroll parameter.	
				\begin{enumerate}
				\item The user indicates his will to change the horizontal scroll.
				\item The system updates plot horizontal scroll.
				\end{enumerate}

	\section{Design and Architecture}

	\section{Implementation Details}

		\begin{comment}
			Potential iterations
			It1. Bluetooth + Architecture prototype
			It2. USB + Log
			It3. Architecture rewriting
			It4. Adding it all and finishing

			For iteration description:
				+ objectives
				+ objectives description
					realization, done?
				+ Use Cases realized
				+ expected and spent time
				+ extra objectives
				+ prologue to next iteration

			Describe risk statuses updates in each iteration?
		\end{comment}

		First two iterations planned so as to reach iOS software functionality, next iterations only drafted because dependant on hw research.

		\subsection{Iteration 1}

			The main objectives for this first iteration were
			\begin{itemize} 
				\item the instruction of the team on Android development, 
				\item the lay out of an initial version of the application architecture and
				\item the implementation of the Bluetooth receiver module.
			\end{itemize}

			Learning of android, product disposable.
			Architecture as a prototype to evolve to (or be substituted by) next versions.
			By the end of the iteration, UC1 realization was complete but not final.
			(Time)Not yet hardware development => full team at this => smaller time
			(ObjD) Bluetooth module fully developed except nice user interface and overall user friendlyness
			(Extra) Log writing v1.

			Results and adaptation of pre-planned it2
			
		\subsection{Iteration 2}

			The main objectives for the second iteration were
			\begin{itemize} 
				\item the development of the USB communication module and
				\item the implementation of the Log visualization module.
			\end{itemize}

			Log writing improved.
			USB module done as USB device, valid for arduino and first msp tests
			Log reading implemented, scrolling and such. Tests results indicate low performance, huge memory requirements.
			With the basic interface, first Prototype with (except usb host == 802.15.4) full functionality implemented. That nice and all. Shown to masters and feedback applied.
			On time ?

			Took a loong break here for hardware development
			% Here ends electroiltes

		\subsection{Iteration 3}
			% This is microlites
			This is the first just-drafted iteration. Starting from the fully functional prototype, architectural and performance fixes were mandatory. Also, given the positive state of the hw research scheduling was done for the rest of the iterations.

			The main objectives for this third iteration were
			\begin{itemize} 
				\item the redesign of the application architecture, 
				\item the achievement of required performance in data management and
				\item the scheduling of the rest of the project time.
			\end{itemize}

			~(Scheduling seems strange as an objective)~

			Architecture redesigned targeting easy adaptation and versatility inside the scope of the application. Redesigned visualization initialization flow.
			Performance increased significantly and memory usage reduced THROUGHOUTLY in realtime view.
			Fixes in modules according to new architecture.
			Talk about scheduling??

		\subsection{Iteration 4}
			Final implementation iteration. Final performance increasing fixes and user interface implementation, as well as user-friendliness globally increased.

			The main objectives for this third iteration were
			\begin{itemize} 
				\item the achievement of required performance in rendering and, 
				\item the implementation of user-friendly interfaces.
			\end{itemize}

			This ends the implementation phase, user-friendlyness could be better but what gives. 
			Performance left 50-50 because it was already ok (30fps not 60fps).
			One iteration left, devoted to testing.

		\subsection{Iteration 5}
			Testing and validation with the real thing.

			Hey, us of the future, I hope everything's ok up there!
			
	\section{Closure}
