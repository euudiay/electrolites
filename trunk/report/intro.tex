\chapter{Intro}
\label{cha:intro}
	\section{Project driver}
		\paragraph{}
		The main motivation for developing this project was the fact that it meant
		the gathering of almost every branch of this career. From its very beginning, this
		work involved both software and hardware development, researching on unknown 
		tools and platforms as well as high and low-level design and programming.
		
		\paragraph{}
		Besides, if successful, it would be likely it could become a real product
		and be useful both in a professional and particular scope, which added a practical
		end for the work to be carried out. Something like this could be made thanks to
		the special features --such as less power consumption and investment-- 802.15.4-compliant
		technology provides which dominant ones lack (Bluetooth, for instance).
		
		\paragraph{}
		In addition, not only developing applications for portable devices but accessories
		are mainstream nowadays; thus, getting in touch with these activities could
		provide us with extra experience at leading edge practices.
		
		\paragraph{}
		However, certain areas of the project would mean working on unprecedent techniques
		--as it will be detailed later on within this section-- and dealing with tools
		which were unknown for us at that moment. Hitches like these could lead to the
		unfulfillment of the project, yet they could also add extra value to it if 
		they were overcome.		
		
	\section{State of the art}
		\begin{itemize}
			\item Proyecto EPFL (pedir referencias a Fran)
			%\item Hace un año: estado de la comunicación android-usb y android-802.15.4 (conexión de dispositivos portátiles con sensores biométricos y no biométricos, costes, consumo, acho de banda, ...)
			\item \emph{Android Accessory Development}\\
				As of May, 2011, there were no easy nor official methods to develop
				accessories capable of communicate with Android running devices. At that
				certain time, the release of the Android Accessory Development Kit (ADK)
				was announced in San Francisco, within the context of Google I/O, developers
				conferenced arranged by Google.
				With the release of that kit, Android project opened itself to the development
				of all kind of new accessories which would add potential and functionalities
				it lacked.
				As well as this kit, the following release of Android 3.1 API version completed
				the accessory ecosystem with the inclusion of directly supported host and device
				USB modes --previous versions of the API did only support accessory mode in a 
				provisional way--.
			\item 802.15.4
			\item Estado del arte en cuanto a estados, Nodos Zigbee
		\end{itemize}
	\section{Main goals}
		\begin{enumerate}
			\item Android <- ecg
			\item \emph{Communication with 802.15.4 emmiters}\\\\
				An initial research was carried out so as to study precedent projects which covered communication
				using this standard with Android devices, obtaining no results though. Therefore, developing and
				incorporating this feature to the project would make it pioneer in this field.\\\\
				However, this sort of communication is not natively supported by any existing Android device
				--not even any other widely known portable computing device--, which leads to the following goal
				for the project to fulfill.\\
			\item \emph{Android accessory development}\\\\
				As it was stated before, Android-powered devices are usually equipped with Bluetooth radio
				modules, yet they lack the capability to communicate with devices which implements other 
				standards. In order to achieve this feature for this sort of device, development of an
				specifically designed accessory is both a required and mandatory task.\\\\
				The aforementioned support Android OS provides for USB device and host modes would allow us to
				obtain data processed by the accessory through an available interface for almost every 
				Android device. In particular, USB host mode would be required so that the Android device 
				[!]was able to power the accessory, hence the restriction of using devices running % were?
				Android version 3.1 or newer.\\\\
				For this goal an USB-capable board equipped with an MSP430 microcontroller was chosen for acting
				as the receiver accessory. More specifically, this microcontroller would be running FreeRTOS,
				which would be accordingly modified for dealing with the USB interface and 802.15.4 
				communication.\\\\
				It is also noteworthy that the usage of a prototyping board and a potential miniaturitasion of
				the previously described board were included into the scope of this objective as well.\\
			\item Aplicación completa
		\end{enumerate}
	\section{Document overview}
		Y tal.
		
		% Ideas Pool
		% ===========
		% En esta sección: Motivación y objetivos principales del proyecto
		

		% Objetivo: desarrollo de un receptor mac 802.15.4 (ZigBee) con conexión 
		% usb para dispositivos android y de una aplicación android para visualización 
		% de datos recibidos de ecg

		% Interesante porque engloba gran parte de lo aprendido en la carrera, habiendo 
		% que desarrollar el proyecto en casi todos los niveles de abstracción, desde diseño 
		% y miniaturización de placas hasta desarrollo basado en un framework de muy alto 
		% nivel como es Android, pasando por prototipado en placa(pasando por arduino) y 
		% desarrollo a nivel de microarquitecturas, en concreto FreeRTOs para procesadores MSP430.

		% Novedad en la comunicación USB entre Android y el MSP sin hardware intermedio, 
		% actuando el dispositivo Android de host para eliminar la fuente de alimentación del MSP, 
		% minimizando al máximo el coste del dispositivo de recepción
		
		% Interés en la utilización del stack ZigBee para maximizar la duración de la batería 
		% del shimmer (nodo delineador). e.g. enviando por bluetooth horas, enviando por shimmer días
		%		+ Una red zigbee es mucho más barata (y factible) que una bluetooth. Repetidores zigbee baratos.

		% Utilidad para el mundo real, que podría ser adoptada en hospitales dado que es 
		% algo que ya se ha estado probando en algunos hospitales y esto sería un importante 
		% impulso para el proyecto
		
		% Relacionado con la aplicación real en hospitales (o uso en domicilio particular) 
		% el reducido tamaño del nodo delineador permite gran portabilidad, incluso llevarlo 
		% encima con objeto de monitorización constante.
		
		% Cubre necesidad profesional, médico o enfermera monitorizando un grupo de pacientes 
		% de tamaño arbitrario con un único dispositivo android; o incluso a larga distancia 
		% recibiendo los datos por internet.
		
		% Desarrollando el uso particular, posible aplicación a particulares que posean dispositivos 
		% táctiles android, para automonitorización en enfermedades en que sea necesario.

		% Proyecto ya existente (a nivel europeo, Lausanne, ...)
		% Aplicación ya desarrollada (dentro del mismo proyecto) para dispositivos iOS, en particular 
		% iPhone, en plan prototipo con grandes limitaciones, poco accesible (instalar manualmente 
		% stack bluetooth, implicando jailbreak), sólo por bluetooth, sin funcionalidad de lectura 
		% de logs. Dispositivos iOS alto coste.
